\section{Thesis goals}\label{sec:thesis-goals}
The main goals of the thesis were set as follows.
Apart from the original goals of the thesis,
which were set in the assignment,
many out of scope and future goals arose during the formalization, research and implementation.
These goals and an overviews how to achieve them are outlined in the section \ref{sec:future-work}.

\subsubsection{Study the state-of-the-art approach to computational tasks scheduling}
This thesis extensively studies and describes state of the art algorithms used for computational task scheduling
and load balancing in section \ref{sec:load-balancing}. 
It also presents technologies used, to solve such problems.
The difference between the state-of-the-art load balancing strategies
and why the thesis proposes and implements the new technology instead of using the existing one
is presented in section \ref{subsec:load-balancing-for-optimization-algorithms}.

\subsubsection{Study various types of optimization problems and approaches and understand their computational needs}
The thesis brings an overview of and describes various optimization problems, techniques and algorithms 
that are being used in the real-life situations to solve diverse industries problems 
in section \ref{sec:optimization-algorithms}.

\subsubsection{Study the distributed scalable architecture and approaches to schedule tasks on such architecture}

The distributed architecture study is presented in section \ref{sec:architecture}.
This section also introduces the overall distributed architecture of the load balancer
including the simulation module.

\subsubsection{Design a scheduling and load-balancing module able to ingest various optimization tasks and schedule them with respect to several criteria}
The complex problem of the load balancing of the optimization tasks is formalized in chapter \ref{ch:problem-formalization}.
Based on the problem formalization,
solution design including the final load balancing algorithm is introduced in chapter \ref{ch:solution-design}.

\subsubsection{Implement the scheduler}
The load balancing module design from chapter \ref{ch:solution-design} is used for 
load balancing module implementation in chapter \ref{ch:implementation}.
The very same chapter contains an overview of the module architecture in the section \ref{sec:architecture}.
The architecture was designed to keep future infrastructure development in mind 
and thus few out of scope future steps were outlined in section \ref{sec:future-work}.

\subsubsection{Evaluate the scheduler on a number of scenarios}
The chapter \ref{ch:experiments} describes the way,
how the scheduler and the load balancing system was tested.
It also describes in section \ref{sec:optimization-algorithms-data} how the runtime data of optimization algorithms were collected.
The next section \ref{sec:simulations} presents the scheduler evaluation.