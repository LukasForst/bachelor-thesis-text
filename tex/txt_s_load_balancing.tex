%%
%% Author: lukas
%% 03.01.2019
%%

\section{Load Balancing}\label{sec:load-balancing}
There will be some info about how should server balance itself.
%TODO State of the art balancing - https://www.ibm.com/support/knowledgecenter/en/SS9H2Y_7.7.0/com.ibm.dp.doc/lbg_loadbalancergroup.html
\begin{itemize}
    \item prioritisation - mainly done by priority queues
    \item handover
    \item instance sizing
    \item algorithms - following are methods used in network balancing -> probably can't be used because we need to manage scheduling
    which is heavy on computer resources like CPU/RAM/IO
    \begin{itemize}
        \item The Least Connection Method
        \item The Round Robin Method
        \item The Least Response Time Method
    \end{itemize}
\end{itemize}
\todo{some stuff about load balancing in general}

In general, load balancing can be classified as either \textit{static} or \textit{dynamic}.

\subsection{Static Load Balancing}\label{subsec:static-load-balancing}
Static load balancing is an approach where system information are provided a priori
and load balancer does not use node performance information of the nodes
\footnote{Execution node - Server executing task which is being scheduled by load balancer.
In our case, this task is solving optimization problem by solver.}, to make distribution decisions.
The performance possibilities and the load of the execution point (or node) are not taken in account
when decision - where to execute current task - is being made.
Then, depending on the performance and load of the nodes, balancing server decides which node will execute task.
When a decision is made, no other interaction with executing node, regarding the current task, is being made.
In other words, once the load is allocated to the execution node, it cannot be transferred to another node.\\
The main disadvantage of this approach is

\subsection{Dynamic Load Balancing}\label{subsec:dynamic-load-balancing}
