%%
%% Author: lukas
%% 03.01.2019
%%

\newpage
\section{Load Balancing}\label{sec:load-balancing}
There will be some info about how should server balance itself.
%TODO State of the art balancing - https://www.ibm.com/support/knowledgecenter/en/SS9H2Y_7.7.0/com.ibm.dp.doc/lbg_loadbalancergroup.html
\begin{itemize}
    \item prioritisation - mainly done by priority queues
    \item handover
    \item instance sizing
    \item algorithms - following are methods used in network balancing -> probably can't be used because we need to manage scheduling
    which is heavy on computer resources like CPU/RAM/IO
\end{itemize}
\todo{some stuff about load balancing in general}
\todo{it would be useful to mention hardware load balancing like routers etc.i}

In general, load balancing can be classified as either \textit{static} or \textit{dynamic}.

\subsection{Static Load Balancing}\label{subsec:static-load-balancing}
Static load balancing is an approach where system information are provided a priori
and load balancer does not use performance information about execution node
\footnote{Execution node - Server executing task which is being scheduled by load balancer.
In our case, this task is solving optimization problem by solver.},
to make distribution decisions.
The performance possibilities and the load of the execution point (or node) are not taken in account
when decision - where to execute current task - is being made, because load-balancing decisions are made at compile time.
When a decision is made, no other interaction with executing node, regarding the current task, is being made.
In other words, once the load is allocated to the execution node, it cannot be transferred to another node.
Static load balancing method is to reduce the overall execution time of a concurrent program while minimizing the communication delays\cite{web:loadBalancingInGridComputing}.
The main advantage of static load balancing methods is mainly the fact, that there is minimal communication delay between system nodes
and therefore execution overhead is minimized to almost zero.
For that reason is static load balancing mainly used in the fields, where server response is crucial such as serving a web page.\todo{Find better example}
Also the implementation of some static load balancing algorithm is straightforwards, since the used methods are very simple.

The main disadvantage of static load balancing is that it does not take in account current state of the system, when making decision.
This could potentially lead to performance issues in the whole system because some nodes can be overloaded although others are not working at all.

Another drawback of this approach is that hardware resources are allocated only once in the execution time.
Since optimization jobs are very heterogeneous, they sometimes have different power requirements during the execution.
For example \textit{TASP} uses only one thread when creating feasible plan in the first algorithm iteration -
this task relays only on single core performance.
However, when first iteration is completed, all following can be done by multiple threads,
therefore it could be useful to execute first iteration on a machine with better single core performance
and then transfer algorithm into machine focused on multiple threads execution.
This is something that can not be done while using static load balancing.\newline
Following static load balancing algorithms are commonly used.

\subsubsection{First Alive}
First alive or also called \textit{Central Manager} algorithm uses the concept of a primary server and backup servers\cite{web:ibmLoadBalancingDecisions}.
All tasks are scheduled to be executed on primary server unless the primary server is down.
Then the load will be forwarded to first backup server.
This algorithm has almost zero level of inner process communication, which leads to better performance when there are lots of smaller tasks.

\subsubsection{Round Robin}
Round Robin algorithm which distributes work load evenly to all nodes.
It is being done in round robin order, where load is distributed to each node in circular order without any priority.
Round Robin is esy to implement and as well as \textit{First alive} algorithm has almost none inner communication overhead.
This algorithm performs best when tasks have equal, or at least similar, processing time.

\subsubsection{Weighted Round Robin}
Weighted round robin algorithm maintains a weighted list of servers and forwards new connections in proportion to the weight, or preference,
of each server.
This algorithm uses more computation times than the round robin algorithm.
However, the additional computation results in distributing the traffic more efficiently to the server that is most capable of handling the request\cite{web:ibmLoadBalancingDecisions}.

\subsubsection{Threshold Algorithm}
Threshold algorithm - execution nodes keep private copy of the system's load, when the load state of a node exceeds a load level limit,
node sends message to all remote nodes, that it is overloaded.
If the local state is not overloaded then the load is allocated locally.
Otherwise a remote node, that is not overloaded, is selected and if no such node exists it is also allocated locally.
This algorithm has low inter process communication and large number of local process allocations.
The later reduces the overhead of remote process allocation and the overhead of remote memory access,
which leads to performance improvements\cite{web:staticAndDynamicLoadBalancing}.

\subsubsection{Least Connections}
Least connections algorithm maintains a record of active server connections
and forward a new connection to the server with the least number of active connections\cite{web:ibmLoadBalancingDecisions}.
This can be generally useful while having many concurrent requests, that can be dispatched quickly.

\subsubsection{Randomized Algorithm}
Randomized algorithm uses random selection of the execution node without having any information about it.

\subsection{Dynamic Load Balancing}\label{subsec:dynamic-load-balancing}
Unlike static load balancing algorithms, dynamic algorithms use runtime state information to more informative decisions while distributing the jobs.
They monitor changes on the system work load and take it in account when decision, where to execute job, is being made.
The process of monitoring the system is not stopped after execution job started and if circumstances change,
job execution can be transferred to another system node which then proceeds with execution.

While many different load balancing algorithms have been proposed, there are four basic steps that nearly all algorithms have in common\cite{malik2000dynamic}.
\begin{enumerate}
    \item Monitoring workstation performance (load monitoring)
    \item Exchanging this information between workstations (synchronization)
    \item Calculating new distributions and making the work movement decision (rebalancing criteria)
    \item Actual data movement (job migration)
\end{enumerate}
Dynamic load balancing algorithms can be divided into two groups based on their control form,
or in other words, where load balancing decisions are made\cite{malik2000dynamic}.
\begin{itemize}
    \item Centralized - a single node in the network is responsible for all load distribution
    \item Distributed - all nodes ale equal
\end{itemize}
While in centralized scheme are decisions made in one master workstation,
in distributed scheme, the load balancing algorithm runs on all nodes
and each node balances itself.
Each of this approach has its own ups and downs,
centralized scheme can be potential performance bottleneck
since it relies on one system node,
on the other hand distributed scheme has communication overhead,
because it requires broadcast communication between all algorithm instances.

The main advantage of dynamic load balancing is that it allows changing execution node in runtime.
For that reason it is possible to change hardware characteristics according to the job execution phase.
For example execute initial phase of optimization algorithm on machine with powerful single core performance
and then move the job to the machine with multiple, less powerful, cores to let it run in parallel.\todo{This sentence sounds weird}
Also as a result of runtime scheduling,
dynamic load balancing algorithms tend to provide a significant improvements in performance over static algorithms.
However, this comes at the additional cost of collecting and maintaining load information\cite{malik2000dynamic}.
For that reason dynamic load balancing suites better for long running tasks, which can be managed and distributed better, than for fast queries.

\subsubsection{Dynamic load balancing strategies}
There are three major parameters which usually define the strategy a specific load balancing algorithm will employ.
These three parameters answer three important questions\cite{malik2000dynamic}:
\begin{enumerate}
    \item Who makes the load balancing decision?
    \item What information is used to make the load balancing decision?
    \item Where the load balancing decision is made?
\end{enumerate}

Question number $1$ is answered based on whether a \textit{sender-initiated} or \textit{receiver-initiated} policy is employed.

Question `What information is used to make the load balancing decision?` is answered by control policies - \textit{global} and \textit{local}.


\bigskip \noindent I would like to present two general dynamic load balancing algorithms - \textit{Central Queue Algorithm} and \textit{Local Queue Algorithm}.

\subsubsection{Central Queue Algorithm}
Central queue algorithm is based on centralized receiver-initiated load balancing strategy.

It uses a cyclic FIFO queue on the main host to store new activities\footnote{Activities - jobs to be executed, in our case optimization job}
and unfulfilled requests.
New activity request is inserted into queue and here it is stored until some execution node picks it up.

Whenever a request for an activity (which is send by executing node in the case when its load has fallen bellow specified threshold)
is received by the queue manager\footnote{Queue manager - central server which manages queue},
it removes the first activity from the queue and sends it to the requester.
If the queue is empty, the request is buffered, until a new activity is available.
If a new activity arrives at the queue manager while there are unanswered requests in the queue,
the first such request is removed from the queue and the new activity is assigned to it.

When a execution node load falls under the threshold,
the local load manager sends a request for a new activity to the central load manager (which manages the central system queue).
The central load manager answers the request immediately if a ready activity is found in the queue,
or queues the request until a new activity arrives\cite{sharma2008performance}.


\subsubsection{Local Queue Algorithm}
Local queue algorithms uses distributed receiver-initiated strategy.

Its main feature is, that it supports dynamic process migration.\todo{Weird sentence}
This algorithm in the first step uses static allocation of all new processes - all processes are allocated to under loaded hosts.
In the second step the process migration is initiated by a host when its load falls under predefined threshold\footnote{
This threshold can be defined by the user and it is an input for the algorithm}.
In such case, the execution node attempts to get several processes from remote hosts.
It randomly sends requests with the number of local ready processes to remote load managers.
When a load manager receives such a request, it compares the local number of ready processes with the received number.
If the former is greater than the latter, then some of the running processes are transferred to the requester
and an affirmative confirmation with the number of processes transferred is returned.\cite{sharma2008performance}

Local queue algorithm is distributed load balancing algorithm where each execution node requests a new activity when it is under loaded.
The main advantage of using such algorithm is the fact, that there is no central point,
where all requests are managed and distributed to another segments of system.
For that reason is this particular algorithm copes and performs well under an increased or expanding workload.

\subsection{Static vs.
Dynamic scheduling}\label{subsec:static-vs.-dynamic-scheduling}
Comparison between static and dynamic scheduling\todo{There must be something about it}.
For our case dynamic is definitely better and more suitable because\ldots