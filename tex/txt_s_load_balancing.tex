%%
%% Author: lukas
%% 03.01.2019
%%

\section{Load Balancing}\label{sec:load-balancing}
There will be some info about how should server balance itself.
%TODO State of the art balancing - https://www.ibm.com/support/knowledgecenter/en/SS9H2Y_7.7.0/com.ibm.dp.doc/lbg_loadbalancergroup.html
\begin{itemize}
    \item prioritisation - mainly done by priority queues
    \item handover
    \item instance sizing
    \item algorithms - following are methods used in network balancing -> probably can't be used because we need to manage scheduling
    which is heavy on computer resources like CPU/RAM/IO
    \begin{itemize}
        \item The Least Connection Method
        \item The Round Robin Method
        \item The Least Response Time Method
    \end{itemize}
\end{itemize}
\todo{some stuff about load balancing in general}

In general, load balancing can be classified as either \textit{static} or \textit{dynamic}.

\subsection{Static Load Balancing}\label{subsec:static-load-balancing}
Static load balancing is an approach where system information are provided a priori
and load balancer does not use performance information about execution node
\footnote{Execution node - Server executing task which is being scheduled by load balancer.
In our case, this task is solving optimization problem by solver.},
to make distribution decisions.
The performance possibilities and the load of the execution point (or node) are not taken in account
when decision - where to execute current task - is being made, because load-balancing decisions are made at compile time.
When a decision is made, no other interaction with executing node, regarding the current task, is being made.
In other words, once the load is allocated to the execution node, it cannot be transferred to another node.
Static load balancing method is to reduce the overall execution time of a concurrent program while minimizing the communication delays.\cite{web:loadBalancingInGridComputing}

The main disadvantage of static load balancing is that it does not take in account current state of the system,
when making decision.
This could potentially lead to performance issues in the whole system because some nodes can be overloaded although others are not working at all.

Another drawback of this approach is that hardware resources are allocated only once in the execution time.
Since optimization jobs are very heterogeneous, they sometimes have different power requirements during the execution.
For example \textbf{TASP} uses only one thread when creating feasible plan in the first algorithm iteration -
this task relays only on single core performance.
However, when first iteration is completed, all following can be done by multiple threads,
therefore it could be useful to execute first iteration on a machine with better single core performance
and then transfer algorithm into machine focused on multiple threads execution.
This is something that can not be done while using static load balancing.\\
Following static load balancing algorithms are commonly used.
\begin{itemize}
    \item \textbf{First alive} or also called \textit{Central Manager} algorithm uses the concept of a primary server and backup servers.
    ~\cite{web:ibmLoadBalancingDecisions}
    All tasks will be scheduled to be executed on primary server unless the primary server is down.
    Then load will be forwarded to first backup server.
    This algorithm has almost zero level of inner process communication, which leads to better performance when there are lots of smaller tasks.

    \item \textbf{Round Robin} algorithm which distributes work load evenly to all nodes.
    It is being done in round robin order, where load is distributed to each node in circular order without any priority.
    Round Robin is esy to implement and as well as \textit{First alive} algorithm has almost none inner communication overhead.
    This algorithm performs best when tasks have equal, or at least similar, processing time.

    \item \textbf{Weighted Round Robin} algorithm maintains a weighted list of servers and forwards new connections in proportion to the weight, or preference,
    of each server.
    This algorithm uses more computation times than the round robin algorithm.
    However, the additional computation results in distributing the traffic more efficiently to the server that is most capable of handling the request.
    ~\cite{web:ibmLoadBalancingDecisions}

    \item \textbf{Threshold algorithm} - execution nodes keep private copy of the system's load, when the load state of a node exceeds a load level limit,
    node sends message to all remote nodes, that it is overloaded.
    If the local state is not overloaded then the load is allocated locally.
    Otherwise a remote node, that is not overloaded, is selected and if no such node exists it is also allocated locally.
    This algorithm has low inter process communication and large number of local process allocations.
    The later reduces the overhead of remote process allocation and the overhead of remote memory access,
    which leads to improvement in performance.\cite{web:qDifferenceBetweenStaticAndDynamicBalancing}

    \item \textbf{Least Connections} algorithm maintains a record of active server connections
    and forward a new connection to the server with the least number of active connections.\cite{web:ibmLoadBalancingDecisions}
    This can be generally useful while having many concurrent requests, that can be dispatched quickly.

    \item \textbf{Randomized algorithm} uses random selection of the execution node without having any information about it.

\end{itemize}

\subsection{Dynamic Load Balancing}\label{subsec:dynamic-load-balancing}
