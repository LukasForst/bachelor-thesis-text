\section{Development stack}\label{sec:development-stack}
Definition of development stack
Add that target environment is JVM 11 but it has backwards compatibility to Java 8.

\subsection{Programming Language}\label{subsec:programming-language}
The OLB is not bound to the single technology, which could limit the development stack,  
and for that reason,
I had a free choice while making the decision about programming language used for the OLB implementation.

OLB is programmed in the next generation programming language \textbf{Kotlin}.
This cross-platform, statically typed, general-purpose programming language is developed by JetBrains\todo{add citations}.
Kotlin is 100\% interoperable with Java because it uses JVM\footnote{Java Virtual Machine - runtime environment for Java byte code} 
as its runtime and it is compiled to the Java Bytecode.
Apart from Java Bytecode, it can be also compiled to JavaScript or native code.\todo{citation needed}
The main advantage of Kotlin is its strong and aggressive type inference,
meaning that for the most of time,
it is not necessary to specify used data type since Kotlin compiler is able to infer it from the context.\todo{citation needed}
It results to concise language syntax and therefore to the faster development in general.

Another great advantage is Kotlin's \textit{null safety}. 
Kotlin compiler distinguishes between non-null types and nullable types 
and enforces \textit{null checks}\footnote{Check whether the object being used has not null value} when the object has nullable data type.
This feature effectively leads to less problems in code (also called \textit{bugs})
and drastically reduces \textit{Null Pointer Exceptions}\footnote{Exception raised when code access reference that has null value}
during the runtime.

\todo{maybe add something about functional approach}

\subsection{Build environment}
As a build automation system I used Gradle, 
which is high performance choice mainly because its incrementally build system,
that works by tracking input and output of tasks, 
including files changes tracking, and only running tasks, that are necessary. 
Also, it processes only these files, that were changed between tasks execution. 
Another reason I choose Gradle was, that it is preferred build system for Kotlin.

To build the application using only Gradle,
it is necessary to have installed at least JVM 8.
There are pre-prepared Gradle wrapper (\textit{gradlew}) scripts,
that are able to build the application without having the Gradle installed on the local machine.

However, the preferred approach to build the application is to use Docker
and build the application to the Docker image, 
which can be then run inside the Docker container.

\subsubsection{Docker build environment}
To keep build clean and reusable on almost every operating system and 
machine setup I decided to use \textbf{multistage Docker build} \todo{add link to docker label}
which uses different base docker images for the build and for the run phase.
Since OLB targets JVM 11 environment and uses Gradle as its build system,
\textit{gradle:5.4.0-jdk11-slim} is used as base image for build stage.
This image contains all necessary Gradle build tools while having smaller size than common Gradle Docker image.
Even smaller (in terms of size) are \textit{alpine} based docker images. 
Alpine is smallest possible Linux core, 
which is widely used in the wide range of Docker base images.
Alpine is focused on the smallest possible size of the image, 
while having all necessary tools build in.
Unfortunately, there were (at the time of development) no official JVM 11 alpine images
since there is no official stable OpenJDK\footnote{Open-source implementation of the Java Platform, Standard Edition} 
11 build for Alpine Linux.

\subsection{Runtime environment}
There are multiple ways, how to start and run the application on the local machine.
\begin{itemize}
	\item As a container inside Docker system - \textbf{preferred}
	\item Locally on JVM 11
	\item Locally on the older JVM (but at least JVM 8)
\end{itemize}

The preferred runtime environment is Docker system, 
where application image runs inside the created docker container,
this is described in the next subsection.\todo{add ref to label}

Although this is preferred execution approach,
there are few other approaches,
how to start and run the application.

It is possible to start the application locally (without Docker environment) by having JVM installed directly on the machine.
Published application setup targets JVM 11, 
therefore for the successful application execution there must be present latest version of JVM 11.

However,
it is also possible to run the application on the previous versions of the JVM up to JVM 8.
Using this way of application execution means,
that it must be build directly under local JVM 8 using the Gradle build system.

\subsubsection{Docker runtime environment}\label{subsubsec:docker-runtime-env}
The build application files are copied from the Docker build stage to the Docker runtime stage.
As the runtime base image in multistage build was used \textit{openjdk:11-jre-slim} image,
because it is official OpenJDK 11 Docker image and therefore it is declared as stable.

Because there was used \textit{gradle application plugin} while building the application, 
startup scripts were generated by the Gradle.
These scripts are then used to start the application itself inside the Docker container.

When starting the whole application, 
multiple services must be started up.
Therefore, because of the containerized environment,
where containers can not access each other,
multiple containers must be started and virtual network connecting them together must be created.
This process can be automated using Docker Compose.

\subsubsection{Docker Compose}
Docker Compose\todo{add link via citation} is an application for defining, running and managing multi-container Docker applications.
It automatically creates Docker networks as well as Docker volumes.\todo{Maybe add more info how it works}
With writing down the definition of multiple Docker applications to the one Docker Compose configuration file,
it is possible to create powerful micro services architecture, 
which can be build or started using single command.

Thanks to the created Docker networks,
containers can communicate with each other using Docker Compose service names,
therefore they do not need to know specific IP address they are running on.

Docker Compose is used in the implementation of OLB since it is designed with micro services architecture in mind.
There are two services - Scheduling server and Scheduling client.
Scheduling server provides ability to schedule process execution on the various computers
and contains all core algorithms.
Scheduling client is an example application which uses ability of scheduling server. 
There are implemented various simulations,
which are being executed by scheduling client.  

\subsection{Framework}
Because of the overall micro services architecture of the project,
some kind of web framework was needed.
There are many Java based web frameworks,
that could be used. 
I would like to present two of them - \textit{Spring Boot}, 
which is traditional and widely used web framework for all kind of Java web applications
and \textit{Ktor} - relatively new, 
lightweight Kotlin framework build upon the Kotlin Coroutines\footnote{Way of asynchronous or non-blocking programming
that generalize subroutines for non-preemptive multitasking with using suspended/resumed task execution}.


\subsubsection{Spring Boot}
Spring Boot\todo{add link via citations} % https://spring.io/projects/spring-boot
is an open source Java Spring-powered web framework.
It takes an opinionated view of the Spring platform,
meaning that Spring Boot automatically configure Spring and 3rd party libraries whenever possible,
and therefore enables usage of it to wider audience.
It is highly dependent on the starter templates feature which provide pre-prepared templates for various types of web applications.
This for example allows user to start with already working we server
and thus simplify the start of the application development.
% cite https://howtodoinjava.com/spring-boot-tutorials/
Spring Boot contains comprehensive infrastructure support for developing enterprise monolith web applications as well as micro services. 
% cite https://github.com/spring-projects/spring-boot/tree/5aeb31700df8e15d1aa625b987ecca93385a7c2c

\subsubsection{Ktor}
Ktor\todo{add link via citation} is an open source web framework for building asynchronous servers 
and clients in connected systems such as web applications and http services.
It designed for quickly building web applications with minimal effort 
and it doesn't impose a lot of technology constraints such as logging, persistent, serializing, dependency injection etc.
It is developed by the same company as Kotlin is, JetBrains.


\bigskip
The final decision was to use \textbf{Ktor} as the web framework,
mainly because of its very light implementation and native Kotlin support.
Also, for such project, 
the features of Spring would not be fully utilized
and therefore the complexity of Spring could potentially slowed down the entire application.

Because Ktor by default does not contain any dependency injection framework, 
I decided to use lightweight DI\footnote{Dependency Injection} framework \textit{Koin}\todo{add link via citations}.
This framework is written in Kotlin and have its own DSL\footnote{Domain Specific Language} for the dependency specification,
which is very handy for the medium sized project.

\subsubsection{Route discovery library}
My developed library for route discovery for KTro

\subsection{Build and deployment}\label{subsec:build-and-deployment}
Info how to build everything and how to deploy it to the AWS for example.
