\chapter{Conclusion}\label{ch:conclusion}
This thesis brings overview of the state of the art solutions for load balancing systems
and proposes, describes and brings implementation of the new method 
of the process of making the load balancing decisions when the process domain of optimization algorithms is well known. 


This thesis focused on the new way,
how to execute load balancing decisions when scheduling  load balancing system
and brings new way of the load balancing on top of the specific optimization jobs domain.
It also proposed and tested the method,
how to make more informative decisions while scheduling optimization algorithms execution in heterogenous system.

The solution with 6767 lines of Kotlin 
code\footnote{Lines provided by \inlinecode{find . -name "*.kt" | xargs cat | wc -l} bash command in root folder of the project} 
proved, that it is possible to use it as load balancing algorithm for optimization algorithms.

\section{Future work}\label{sec:future-work}
To achieve production ready load balancing system, 
an execution module, 
which would transform the created plan into physical actions 
(like running, stopping and moving the jobs between the physical execution nodes)
performed in the infrastructure,
is needed.

\subsubsection{Infrastructure development}
As soon as the previously mentioned execution module is implemented,
the application can be deployed into real life environment.
Proposition is to make optimization algorithms running as Docker containers 
and the execution module would operate with Docker machines,
which would be running on the execution nodes. 
In this way, 
the assigned resources would be very easily changed on one execution node.
Migration to the next execution node would not be problem as well, 
since containers could be wrapped and transported through the network.

Another infrastructure related missing feature in current solution is full REST API.
Currently only binary serialization is supported.
In the future, 
full REST with JSON as its transport format should be supported.

\subsubsection{Routes discovery library}
Routes discovery library was very handy during the development of application's server parts 
and I believe that this way of routes registration in Ktor would suit to many developers as well.

Therefore I would like to refactor it from the base project 
and create open source project which will ensure future library development.
I would like to also make it more generic,
because right know, 
it depends on specific Ktor and Koin version.
Although Ktor dependency is necessary
since it is library developed specifically for Ktor,
Koin should be replaced by generic way, 
how to obtain dependencies for routes.

I believe in future usage and development of the library
and I hope, 
that there will be many future developers,
that will agree with me.


\subsubsection{Extension functions}
During the application development,
I created and tested many Kotlin extension functions.
These functions are mainly not domain specific
and for that reason I decided that it would be fine to publish them as well as Routes discovery library.
These extensions could be useful when starting new project,
because they are able to perform many operations in single line of Kotlin code.