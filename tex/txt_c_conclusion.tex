\chapter{Conclusion}\label{ch:conclusion}
This thesis focused mainly on the core of the proposed load balancing system
and outlined the way,
how to make more informative decisions while scheduling optimization algorithms execution in heterogenous system.

The solution with 6767 lines of Kotlin 
code\footnote{Lines provided by \inlinecode{find . -name "*.kt" | xargs cat | wc -l} bash command in root folder of the project} 
proved, that it is possible to use it as load balancing algorithm for optimization algorithms.
Although the core part of the scheduling server for optimization algorithm was developed in this thesis,
there is lot of work being left.

\section{Future work}\label{sec:future-work}
The main missing part is an execution module, 
which would transform the created plan into physical actions performed in the infrastructure 
like running, stopping and moving the jobs between the physical execution nodes.

\subsubsection{Infrastructure development}
As soon as the previously mentioned execution module is implemented,
the application can be deployed into real life environment and properly tested.
Proposition is to make optimization algorithms running as Docker containers 
and the execution module would operate with Docker machines,
which would be running on the execution nodes. 
In this way, 
the assigned resources would be very easily changed on one execution node.
Migration to the next execution node would not be problem as well, 
since containers could be wrapped and transported through the network.

\subsubsection{Documentation and unit testing}
Another thing, that must be improved, is documentation as well as unit testing of the application.
Unfortunately, the current lack of code documentation could discourage from future development of the system.
As for the unit tests, 
only crucial parts of application such as extension functions and prediction module are at least partially unit tested.
The goal is to have documented and tested whole core module,
which is responsible for the scheduling itself and public APIs of server modules. 

\subsubsection{Routes discovery library}
Routes discovery library was very handy during the development of application's server parts 
and I believe that this way of routes registration in Ktor would suit to many developers as well.

Therefore I would like to refactor it from the base project 
and create open source project which will ensure future library development.
I would like to also make it more generic,
because right know, 
it depends on specific Ktor and Koin version.
Although Ktor dependency is necessary
since it is library developed specifically for Ktor,
Koin should be replaced by generic way, 
how to obtain dependencies for routes.

I believe in future usage and development of the library
and I hope, 
that there will be many future developers,
that will agree with me.


\subsubsection{Extension functions}
During the application development,
I created and tested many Kotlin extension functions.
These functions are mainly not domain specific
and for that reason I decided that it would be fine to publish them as well as Routes discovery library.
These extensions could be useful when starting new project,
because they are able to perform many operations in single line of Kotlin code.