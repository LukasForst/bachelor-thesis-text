\documentclass[thesis=B,english]{BPthesis}[2012/06/26]

% basic packages
\usepackage[utf8]{inputenc} % input in utf8
\usepackage{cite} % bibtex
\usepackage{url} % so URLs look better in bib
\usepackage{amsmath} % math extensions

% fancy packages
\usepackage{indentfirst} % indent every time after new line
\usepackage{xcolor} % various colors used in listings and in template

\usepackage[square,numbers]{natbib} % bibliography style with numbers
\bibliographystyle{abbrvnat}

\hypersetup{hidelinks=true} % disable hideous link borders

% specific packages
\usepackage[linesnumbered,boxed]{algorithm2e} % algorithm engine, lines are numbered and it is in box
\SetKwInput{Input}{Input} % set keyword for algorithm2e
\SetKwInput{Output}{Output} % set keyword for algorithm2e

\usepackage{listings} % programming examples
\lstdefinelanguage{Kotlin}{ % Kotlin code in the document
  comment=[l]{//},
  commentstyle={\color{gray}\ttfamily},
  emph={delegate, filter, first, firstOrNull, forEach, lazy, map, mapNotNull, println, return@},
  emphstyle={\color{black}},
  identifierstyle=\color{black},
  keywords={abstract, actual, as, as?, break, by, class, companion, continue, data, do, dynamic, else, enum, expect, false, final, for, fun, get, if, import, in, interface, internal, is, null, object, override, package, private, public, return, set, super, suspend, this, throw, true, try, typealias, val, var, vararg, when, where, while},
  keywordstyle={\color[RGB]{0 0 128}\bfseries},
  morecomment=[s]{/*}{*/},
  morestring=[b]",
  morestring=[s]{"""*}{*"""},
  ndkeywords={@Route, @Deprecated, @JvmField, @JvmName, @JvmOverloads, @JvmStatic, @JvmSynthetic, Array, Byte, Double, Float, Int, Integer, Iterable, Long, Runnable, Short, String},
  ndkeywordstyle={\color[RGB]{0 0 128}\bfseries},
  sensitive=true,
  stringstyle={\color[RGB]{0 128 0}\ttfamily},
  backgroundcolor=\color[RGB]{244 244 244}
}

\newcommand{\inlinecode}[1]{\colorbox[RGB]{244,244,244}{\lstinline[language=Kotlin]$#1$}} % one line kotlin code
\newcommand{\inlinedata}[1]{\colorbox[RGB]{244,244,244}{\texttt{#1}}} % one line of data

% TODO remove when ready
% \usepackage[disable]{todonotes} % don't generate todos to pdf
\usepackage{todonotes} % show todos in generated pdf

%template setup
\department{Department of Computer Science}
\title{Design and implementation of a scalable server for parallelization of optimization algorithms execution}
\authorGN{Lukáš}
\authorFN{Forst} 
\authorWithDegrees{Lukáš Forst}
\author{Lukáš Forst}
\supervisor{Ing. Ondřej Vaněk, Ph.D.}
\placeForDeclarationOfAuthenticity{Prague}
\declarationOfAuthenticityOption{1} % 1-6
\keywordsCS{vyvažování zátěže, optimalizační algoritmy, Kotlin, Ktor, Docker}
\keywordsEN{load balancing, optimization algorithms, Kotlin, Ktor, Docker}   

\acknowledgements{
  \noindent
  I would first like to thank my thesis advisor Ondřej Vaněk,
  whose ofice's door was always open whenever I ran into a trouble spot or had a question about my research or writing. 
  
  \noindent
  I would like to also thank to my colleagues in Blindspot Solutions and especially to Petr Eichler,
  whose valuable comment suggestions on my paper gave me an inspiration to improve the quality of the assignment
  and helped me to overcome various challenges I faced furing the development.
  
  \noindent
  Finally, I must express my very profound gratitude to my parents and to my better half for providing me with unfailing support 
  and continuous encouragement throughout my years of study and through the process of researching and writing this thesis. 

  \noindent
  This accomplishment would not have been possible without them.
  
  \bigskip \noindent
  Thank you.
  
  \bigskip \noindent
  Lukáš Forst
}

\abstractCS{
	abstract cs\todo{czech abstract here}
}
\abstractEN{
	abstract en\todo{en abstract there}
}

\begin{document}
    
%%
%% Author: lukas
%% 01.01.2019
%%

\chapter{Introduction}\label{ch:introduction}
\todo{Whole text does not seems to be right, maybe I will need to rewrite it.}
Optimization algorithms and solutions build on them are widely used in current manufacturing industry to reduce production costs.
With more and more production automatization, optimization algorithms can manage and schedule whole factories with maximum available efficiency.

Complexity of optimization problems could be huge and therefore performance requirements are sometimes not easily satisfiable.
Using one powerful instance of optimization algorithm in cloud seems like a solution for problems with smaller complexity,
but what if we have multiple huge problems where each is performance demanding?
Of course, we can create multiple instances, but that would be expensive and not well manageable and scalable
since adding another instances manually requires some time and it is not much flexible.
Another disadvantage of this approach is the fact, that optimization algorithm is not running $100\%$ of time
and thus resources allocated by this algorithm are unused while other algorithm instances could be potentially overwhelmed.
Also paying for unused hardware is wasting money and optimization algorithms are supposed to save money.

Now imagine having two completely different problems that each requires its own application which visualises data
and optimization algorithm to compute some kind of plan,
this algorithm can be generic enough to operate on both domains with same code base, but it requires a lot of performance resources.
If we use monolithic architecture of both applications,
we would have same code in two applications,
but what is even worse, we would need two powerful machines to run our applications.
As previously mentioned, these two machines would not be using their power whole time and would be mainly idle.

What if one application runs only few minutes a day, but needs that power to complete tasks in time?
A lot of resources would be wasted if it has its own server,
but using not powerful server would lead to increasing duration of ongoing tasks which is something we do not want.

In this paper I would like to introduce \textbf{load balancer} specifically developed for optimization algorithms
which could potentially minimize resources wasting and increase performance using correct utilization distribution across
multiple instances of optimization algorithms.


%%
%% Author: lukas
%% 03.01.2019
%%

\chapter{Problem definition}\label{ch:problem-definition}

The problem with implementation of optimization algorithms in applications is that their
performance requirements are quite high and are fully utilized only while working.
Optimization algorithm is not running all the time and for that reason hardware resources are mainly unused.
These unused resources could be potentially used by another instance of algorithm
or can be shutdown completely to reduce hosting costs.

Also adding more time to job execution does not always bring better solution
but it certainly costs more.
Therefore proposed load balancer must be able to stop execution when solution value
is not getting better compared with scheduling costs.

%%
%% Author: lukas
%% 12.02.2019
%%

\section{Formal definition}\label{sec:formal-definition}
Whole definition implies that it can be represented as Integer Linear Programming (ILP).
However, it is not entirely formalized, 
it is only mechanical work to transcript all implications into equations.
For that reason, and better readability, I left the implications.

\subsection{Variables Definition}\label{subsec:variables-definition}
Following indexes, inputs and variables are used in the optimization criteria.

\subsubsection{Indexes}
\begin{itemize}
	\item $j$ - index used to identify something related to the execution job,
	      in real world this is most likely job id, located in the right upper corner - $x^{j}$, 
	      set of all jobs in the system is represented by $J$
	\item $r$ - index used to identify resources, written in the left upper corner - ${}^{r}x$, set of resources is represented by $R$
	\item $t$ - right bottom index represents time - $x_t$
\end{itemize}

\subsubsection{Input}\label{subsubsec:formal-input}
Input which is specified before executing optimization job by the user outside of the system.
When new execution job is requested - this is done by the user, or client application,
following data should be provided.
\begin{itemize}
	\item $D^{j}$ - maximal duration of the job execution which cannot be exceeded
	\item $P^{j}$ - maximal used resources cost per job, or in other words highest possible price paid for the job execution which cannot be exceeded
\end{itemize} 
There are also constants defined before load balancer start up.
\begin{itemize}
	\item ${}^{r}c$ - cost of the using particular resources per one time unit
\end{itemize}
Each of the previously mentioned variable must be non-negative.

\subsubsection{Program Output}
Apart from result of the underlying optimization algorithm,
following data are returned to user after successful job execution.

\begin{itemize}
	\item $T^{j}$ - time taken, duration of the actual job execution
	\item $C^{j}$ - resource costs, how much money was actually paid for the job execution
\end{itemize}

\subsubsection{Variables}

\begin{itemize}
	\item $v_{t}^{j}$ - value\todo{better example would be fine} (i.e.\ cost of the scheduled plan) of the job $j$ at the time $t$
	      Value is greater than zero and it is non-increasing during the time. 
	      \begin{equation}
	      	\forall t, j: v_{t}^{j} \geq v_{t+1}^{j} > 0 
	      \end{equation}
	      It is non-increasing because optimization algorithms return always best found solution, 
	      so when worse solution, than currently best one, is found,
	      returned is still the best solution found.
	\item $^{r}x_{t}^{j}$ - represents assignment of the resources $r$ at time $t$ to job $j$
	      \begin{equation}
	      	^{r}x_{t}^{j} = \{0, 1\} 
	      \end{equation}
	      Each $x$ is either $1$ = indexed resources are assigned to the job at given time or $0$ = given combination does not have assignment.
	      We assume that each job has only one such assignment at one time,
	      which effectively means that this job is executed on the single computation node.
	      This is defined by following constraint:
	      \begin{equation}
	      	\forall j, t:\; \sum_{r \in R} {}^{r}x_{t}^{j} \leq 1 
	      \end{equation}
	\item $^{r}\Delta_{t}^{j}$ - enhancement of the value $v$ with resources $r$ on the job $j$ per time $t$.
	      \begin{equation}
	      	^{r}\Delta_{t}^{j} = {}^{r}| v_{t}^{j} - v_{t-1}^{j}| \cdot {}^{r}x_{t}^{j}
	      \end{equation}
	      It is improvement of the solution value $v$ which can be achieved by using resources $r$ at time $t$.
	      This value is always non-negative since optimization algorithm always stores best found solution,
		  and therefore $\forall j, t, r:\; {}^{r}\Delta_{t}^{j} \geq 0$. 
		  ${}^{r}x_{t}^{j}$ ensures that only resources, 
		  that are actually used, are taken in account.
	\item $S_{t}^{j}$ - reward for improving solution value until time $t$ per job $j$.
	      Accumulation of enhancements $^{r}\Delta_{t}^{j}$ through all resources $r$ and time units $t$.
	      \begin{equation}
	      	S_{t}^{j} = \sum_{t}\:\sum_{r \in R}\: {}^{r}\Delta_{t}^{j} 
	      \end{equation}
	\item $C_{t}^{j}$ - defines how much execution of job $j$ cost from the beginning of the execution until time $t$.
	      Sum of all allocated resources for their time for the particular job.
	      \begin{equation}
	      	C_{t}^{j} = \sum_{t}\:\sum_{r \in R}\: {}^{r}c \cdot {}^{r}x_{t}^{j} 
	      \end{equation}
	      Because $C_{t}^{j}$ is defined as sum and ${}^{r}c$ is non-negative,
	      it is true that $\forall j, t:\; C_{t+1}^{j} \geq C_{t}^{j}$.
	      Input of the program specifies maximal cost paid for job execution as a $P^j$, 
	      therefore it must be enforced by the system that this cost will not be exceeded.
	      This constraint can be defined as follows.
	      \begin{equation}
	      	\forall t, j:\; P^{j} \leq C_{t}^{j} \implies \sum_{t+1}^{\infty} \, \sum_{r \in R} {}^{r}x_{t}^{j} = 0 
	      \end{equation}
	      Which effectively means that when cost of job execution $C_{t}^{j}$ has reached maximal defined cost $P^{j}$,
	      no resources can be assigned to this job.
	\item $t$ - time, it is not only index but also variable, there are also constraints regarding time -
	      since client application can specify deadline to job $D^{j}$,
	      there must be additional constraint for job execution in a matter of resources assignment.
	      \begin{equation}
	      	\forall t, j:\; D^{j} \leq t \implies \sum_{t+1}^{\infty} \, \sum_{r \in R} {}^{r}x_{t}^{j} = 0 
	      \end{equation}
	      When maximal time is exceeded, no additional resources can be assigned to the job execution, 
	      which could be defined by following constraint.
\end{itemize}

\subsection{Resources reconfiguration}\label{subsec:resource-config}
System should be capable of changing resources assignment per job in the runtime.
This will help to distribute performance according to the current nodes load across whole network.
Unfortunately, it is not always possible to reconfigure resources assignment while scheduling is being performed.
Therefore there must be at least one time unit, between different resources assignment,
where no resources are assigned to the job.
In other words, if resources reconfiguration is triggered in time $t$ then $\sum_{r \in R} {}^{r}x_{t}^{j} = 0$.
This can be formalized as:
\begin{equation}
	\sum_{r \in R} {}^{r}x_{t}^{j} = 1 \implies \forall r \in R:\; {}^{r}x_{t}^{j} - {}^{r}x_{t + 1}^{j} \geq 0
\end{equation}

\subsection{Optimization criteria}\label{subsec:optimization-criteria}
The main goal of the system is to minimize outcome value of the underlining optimization algorithm
and at the same time to minimize cost of used resources.
We can optimize single job or sum of outcomes from all jobs in the system at once.
First approach provides possibility to control and optimize outcome of a particular job,
which is an advantage for single client (job owner),
but it does not necessarily means that it is optimal for the whole system and vice versa.
Optimization criteria for the single job at particular time $t$ is then
maximization of weighted difference between the value enhancement reward 
and cost paid for the enhancement,
which can be described by following equation.
\begin{equation}
	\max crit_{t}^{j} = \alpha S_{t}^{j} - (1 - \alpha) C_{t}^{j} \qquad 0 \leq \alpha \leq 1 
\end{equation}
For optimization of system-wide resources and costs,
all jobs execution optimization is then defined like a
weighted sum of all rewards per jobs lowered by sum of all resources costs across the set of all jobs.
\begin{equation}
	\max crit_{t} = \alpha \sum_{j \in J} S_{t}^{j} - (1 - \alpha) \sum_{j \in J} C_{t}^{j} \qquad 0 \leq \alpha \leq 1 
\end{equation}
Based on the previous equation, 
it is possible to define time independent optimization criterion.
\begin{equation}
	\max crit = \sum_{t}^{\infty} crit_{t}
\end{equation}

%%
%% Author: lukas
%% 03.01.2019
%%

\section{Motivation to solve it}\label{sec:motivation-to-solve-it}
Here comes some motivation to solve it
\todo{Maybe bachelor degree?
Or at least unassisted project would be fine\ldots}
\todo{Lost in the context\ldots}

%TODO how we can save costs

%%
%% Author: lukas
%% 03.01.2019
%%

\section{Motivation to solve it}\label{sec:motivation-to-solve-it}
Here comes some motivation to solve it
\todo{Maybe bachelor degree?
Or at least unassisted project would be fine\ldots}
\todo{Lost in the context\ldots}

%TODO how we can save costs



%%
%% Author: lukas
%% 03.01.2019
%%

\chapter{Problem definition}\label{ch:problem-definition}

The problem with implementation of optimization algorithms in applications is that their
performance requirements are quite high and are fully utilized only while working.
Optimization algorithm is not running all the time and for that reason hardware resources are mainly unused.
These unused resources could be potentially used by another instance of algorithm
or can be shutdown completely to reduce hosting costs.

Also adding more time to job execution does not always bring better solution
but it certainly costs more.
Therefore proposed load balancer must be able to stop execution when solution value
is not getting better compared with scheduling costs.

%%
%% Author: lukas
%% 12.02.2019
%%

\section{Formal definition}\label{sec:formal-definition}
Whole definition implies that it can be represented as Integer Linear Programming (ILP).
However, it is not entirely formalized, 
it is only mechanical work to transcript all implications into equations.
For that reason, and better readability, I left the implications.

\subsection{Variables Definition}\label{subsec:variables-definition}
Following indexes, inputs and variables are used in the optimization criteria.

\subsubsection{Indexes}
\begin{itemize}
	\item $j$ - index used to identify something related to the execution job,
	      in real world this is most likely job id, located in the right upper corner - $x^{j}$, 
	      set of all jobs in the system is represented by $J$
	\item $r$ - index used to identify resources, written in the left upper corner - ${}^{r}x$, set of resources is represented by $R$
	\item $t$ - right bottom index represents time - $x_t$
\end{itemize}

\subsubsection{Input}\label{subsubsec:formal-input}
Input which is specified before executing optimization job by the user outside of the system.
When new execution job is requested - this is done by the user, or client application,
following data should be provided.
\begin{itemize}
	\item $D^{j}$ - maximal duration of the job execution which cannot be exceeded
	\item $P^{j}$ - maximal used resources cost per job, or in other words highest possible price paid for the job execution which cannot be exceeded
\end{itemize} 
There are also constants defined before load balancer start up.
\begin{itemize}
	\item ${}^{r}c$ - cost of the using particular resources per one time unit
\end{itemize}
Each of the previously mentioned variable must be non-negative.

\subsubsection{Program Output}
Apart from result of the underlying optimization algorithm,
following data are returned to user after successful job execution.

\begin{itemize}
	\item $T^{j}$ - time taken, duration of the actual job execution
	\item $C^{j}$ - resource costs, how much money was actually paid for the job execution
\end{itemize}

\subsubsection{Variables}

\begin{itemize}
	\item $v_{t}^{j}$ - value\todo{better example would be fine} (i.e.\ cost of the scheduled plan) of the job $j$ at the time $t$
	      Value is greater than zero and it is non-increasing during the time. 
	      \begin{equation}
	      	\forall t, j: v_{t}^{j} \geq v_{t+1}^{j} > 0 
	      \end{equation}
	      It is non-increasing because optimization algorithms return always best found solution, 
	      so when worse solution, than currently best one, is found,
	      returned is still the best solution found.
	\item $^{r}x_{t}^{j}$ - represents assignment of the resources $r$ at time $t$ to job $j$
	      \begin{equation}
	      	^{r}x_{t}^{j} = \{0, 1\} 
	      \end{equation}
	      Each $x$ is either $1$ = indexed resources are assigned to the job at given time or $0$ = given combination does not have assignment.
	      We assume that each job has only one such assignment at one time,
	      which effectively means that this job is executed on the single computation node.
	      This is defined by following constraint:
	      \begin{equation}
	      	\forall j, t:\; \sum_{r \in R} {}^{r}x_{t}^{j} \leq 1 
	      \end{equation}
	\item $^{r}\Delta_{t}^{j}$ - enhancement of the value $v$ with resources $r$ on the job $j$ per time $t$.
	      \begin{equation}
	      	^{r}\Delta_{t}^{j} = {}^{r}| v_{t}^{j} - v_{t-1}^{j}| \cdot {}^{r}x_{t}^{j}
	      \end{equation}
	      It is improvement of the solution value $v$ which can be achieved by using resources $r$ at time $t$.
	      This value is always non-negative since optimization algorithm always stores best found solution,
		  and therefore $\forall j, t, r:\; {}^{r}\Delta_{t}^{j} \geq 0$. 
		  ${}^{r}x_{t}^{j}$ ensures that only resources, 
		  that are actually used, are taken in account.
	\item $S_{t}^{j}$ - reward for improving solution value until time $t$ per job $j$.
	      Accumulation of enhancements $^{r}\Delta_{t}^{j}$ through all resources $r$ and time units $t$.
	      \begin{equation}
	      	S_{t}^{j} = \sum_{t}\:\sum_{r \in R}\: {}^{r}\Delta_{t}^{j} 
	      \end{equation}
	\item $C_{t}^{j}$ - defines how much execution of job $j$ cost from the beginning of the execution until time $t$.
	      Sum of all allocated resources for their time for the particular job.
	      \begin{equation}
	      	C_{t}^{j} = \sum_{t}\:\sum_{r \in R}\: {}^{r}c \cdot {}^{r}x_{t}^{j} 
	      \end{equation}
	      Because $C_{t}^{j}$ is defined as sum and ${}^{r}c$ is non-negative,
	      it is true that $\forall j, t:\; C_{t+1}^{j} \geq C_{t}^{j}$.
	      Input of the program specifies maximal cost paid for job execution as a $P^j$, 
	      therefore it must be enforced by the system that this cost will not be exceeded.
	      This constraint can be defined as follows.
	      \begin{equation}
	      	\forall t, j:\; P^{j} \leq C_{t}^{j} \implies \sum_{t+1}^{\infty} \, \sum_{r \in R} {}^{r}x_{t}^{j} = 0 
	      \end{equation}
	      Which effectively means that when cost of job execution $C_{t}^{j}$ has reached maximal defined cost $P^{j}$,
	      no resources can be assigned to this job.
	\item $t$ - time, it is not only index but also variable, there are also constraints regarding time -
	      since client application can specify deadline to job $D^{j}$,
	      there must be additional constraint for job execution in a matter of resources assignment.
	      \begin{equation}
	      	\forall t, j:\; D^{j} \leq t \implies \sum_{t+1}^{\infty} \, \sum_{r \in R} {}^{r}x_{t}^{j} = 0 
	      \end{equation}
	      When maximal time is exceeded, no additional resources can be assigned to the job execution, 
	      which could be defined by following constraint.
\end{itemize}

\subsection{Resources reconfiguration}\label{subsec:resource-config}
System should be capable of changing resources assignment per job in the runtime.
This will help to distribute performance according to the current nodes load across whole network.
Unfortunately, it is not always possible to reconfigure resources assignment while scheduling is being performed.
Therefore there must be at least one time unit, between different resources assignment,
where no resources are assigned to the job.
In other words, if resources reconfiguration is triggered in time $t$ then $\sum_{r \in R} {}^{r}x_{t}^{j} = 0$.
This can be formalized as:
\begin{equation}
	\sum_{r \in R} {}^{r}x_{t}^{j} = 1 \implies \forall r \in R:\; {}^{r}x_{t}^{j} - {}^{r}x_{t + 1}^{j} \geq 0
\end{equation}

\subsection{Optimization criteria}\label{subsec:optimization-criteria}
The main goal of the system is to minimize outcome value of the underlining optimization algorithm
and at the same time to minimize cost of used resources.
We can optimize single job or sum of outcomes from all jobs in the system at once.
First approach provides possibility to control and optimize outcome of a particular job,
which is an advantage for single client (job owner),
but it does not necessarily means that it is optimal for the whole system and vice versa.
Optimization criteria for the single job at particular time $t$ is then
maximization of weighted difference between the value enhancement reward 
and cost paid for the enhancement,
which can be described by following equation.
\begin{equation}
	\max crit_{t}^{j} = \alpha S_{t}^{j} - (1 - \alpha) C_{t}^{j} \qquad 0 \leq \alpha \leq 1 
\end{equation}
For optimization of system-wide resources and costs,
all jobs execution optimization is then defined like a
weighted sum of all rewards per jobs lowered by sum of all resources costs across the set of all jobs.
\begin{equation}
	\max crit_{t} = \alpha \sum_{j \in J} S_{t}^{j} - (1 - \alpha) \sum_{j \in J} C_{t}^{j} \qquad 0 \leq \alpha \leq 1 
\end{equation}
Based on the previous equation, 
it is possible to define time independent optimization criterion.
\begin{equation}
	\max crit = \sum_{t}^{\infty} crit_{t}
\end{equation}

%%
%% Author: lukas
%% 03.01.2019
%%

\section{Motivation to solve it}\label{sec:motivation-to-solve-it}
Here comes some motivation to solve it
\todo{Maybe bachelor degree?
Or at least unassisted project would be fine\ldots}
\todo{Lost in the context\ldots}

%TODO how we can save costs

%%
%% Author: lukas
%% 13.01.2019
%%

\chapter{State of the art}\label{ch:state-of-the-art}

%%
%% Author: lukas
%% 03.01.2019
%%

\section{Load Balancing}\label{sec:load-balancing}
Load balancing is technique for a division of processing work in the distributed environment of execution units
\footnote{In general, execution unit can be CPU, network links, storage devices or other devices,
in this paper \textit{execution unit} or also referred as \textit{execution node} or as \textit{host} is a computer executing assigned job}
with aim to deliver faster service with higher efficiency.
It improves the distribution of workloads across the whole environment and thus balances resources usage while maximizing throughput and minimizing response time.
Load balancer is typically either dedicated \textit{hardware device} or \textit{software program}.

A \textbf{hardware} load balancer is a dedicated hardware device which distributes network traffic across a cluster of servers\cite{web:hardwareLoadBalancer}.
These devices are used mainly in the data centers to ensure equal distribution of traffic between the application servers.
Main benefit of using hardware load balancer is zero balancing overhead on the host machines,
because all decisions are made on dedicated hardware specially developed for such tasks.

A \textbf{software} load balancer is a program operating on the application server with the same aim as hardware load balancer.
Main advantage of the software load balancing is that it can be heavily customized and deployed to its own server.
This paper will discuss only software load balancing approach.

\medskip
\noindent In general, software load balancing algorithms can be classified as either \textit{static} or \textit{dynamic}.

\subsection{Static Load Balancing}\label{subsec:static-load-balancing}
Static load balancing is an approach where system information are provided a priori
and load balancer does not use performance information about execution node
\footnote{Execution node - Server executing task which is being scheduled by load balancer.
In our case, this task is solving optimization problem by solver.},
to make distribution decisions.
The performance possibilities and the load of the execution point (or node) are not taken in account
when decision - where to execute current task - is being made, because load-balancing decisions are made at compile time.
When a decision is made, no other interaction with executing node, regarding the current task, is being made.
In other words, once the load is allocated to the execution node, it cannot be transferred to another node.
Static load balancing method is to reduce the overall execution time of a concurrent program while minimizing the communication delays\cite{web:loadBalancingInGridComputing}.
The main advantage of static load balancing methods is mainly the fact, that there is minimal communication delay between system nodes
and therefore execution overhead is minimized to almost zero.
For that reason is static load balancing mainly used in the fields, where server response is crucial such as serving a web page.\todo{Find better example}
Also the implementation of some static load balancing algorithm is straightforwards, since the used methods are very simple.

The main disadvantage of static load balancing is that it does not take in account current state of the system, when making decision.
This could potentially lead to performance issues in the whole system because some nodes can be overloaded although others are not working at all.

Another drawback of this approach is that hardware resources are allocated only once in the execution time.
Since optimization jobs are very heterogeneous, they sometimes have different power requirements during the execution.
For example \textit{TASP}\footnote{\textbf{Task and Asset Scheduling Platform} - proprietary optimization software developed by Blindspot Solutions, described in\ref{subsubsec:tasp}}
uses only one thread when creating feasible plan in the first algorithm iteration - this task relays only on single core performance.
However, when first iteration is completed, all following can be done by multiple threads,
therefore it could be useful to execute first iteration on a machine with better single core performance
and then transfer algorithm into machine focused on multiple threads execution.
This is something that can not be done while using static load balancing.\newline
Following static load balancing algorithms are commonly used.

\subsubsection{First Alive}
First alive or also called \textit{Central Manager} algorithm uses the concept of a primary server and backup servers\cite{web:ibmLoadBalancingDecisions}.
All tasks are scheduled to be executed on primary server unless the primary server is down.
Then the load will be forwarded to first backup server.
This algorithm has almost zero level of inner process communication, which leads to better performance when there are lots of smaller tasks.

\subsubsection{Round Robin}
Round Robin algorithm which distributes work load evenly to all nodes.
It is being done in round robin order, where load is distributed to each node in circular order without any priority.
Round Robin is esy to implement and as well as \textit{First alive} algorithm has almost none inner communication overhead.
This algorithm performs best when tasks have equal, or at least similar, processing time.

\subsubsection{Weighted Round Robin}
Weighted round robin algorithm maintains a weighted list of servers and forwards new connections in proportion to the weight, or preference,
of each server.
This algorithm uses more computation times than the round robin algorithm.
However, the additional computation results in distributing the traffic more efficiently to the server that is most capable of handling the request\cite{web:ibmLoadBalancingDecisions}.

\subsubsection{Threshold Algorithm}
Threshold algorithm - execution nodes keep private copy of the system's load, when the load state of a node exceeds a load level limit,
node sends message to all remote nodes, that it is overloaded.
If the local state is not overloaded then the load is allocated locally.
Otherwise a remote node, that is not overloaded, is selected and if no such node exists it is also allocated locally.
This algorithm has low inter process communication and large number of local process allocations.
The later reduces the overhead of remote process allocation and the overhead of remote memory access,
which leads to performance improvements\cite{web:staticAndDynamicLoadBalancing}.

\subsubsection{Least Connections}
Least connections algorithm maintains a record of active server connections
and forward a new connection to the server with the least number of active connections\cite{web:ibmLoadBalancingDecisions}.
This can be generally useful while having many concurrent requests, that can be dispatched quickly.

\subsubsection{Randomized Algorithm}
Randomized algorithm uses random selection of the execution node without having any information about it.

\subsection{Dynamic Load Balancing}\label{subsec:dynamic-load-balancing}
Unlike static load balancing algorithms, dynamic algorithms use runtime state information to more informative decisions while distributing the jobs.
They monitor changes on the system work load and take it in account when decision, where to execute job, is being made.
The process of monitoring the system is not stopped after execution job started and if circumstances change,
job execution can be transferred to another system node which then proceeds with execution.

While many different load balancing algorithms have been proposed, there are four basic steps that nearly all algorithms have in common\cite{malik2000dynamic}.
\begin{enumerate}
    \item Monitoring workstation performance (load monitoring)
    \item Exchanging this information between workstations (synchronization)
    \item Calculating new distributions and making the work movement decision (rebalancing criteria)
    \item Actual data movement (job migration)
\end{enumerate}
Dynamic load balancing algorithms can be divided into two groups based on their control form,
or in other words, where load balancing decisions are made\cite{malik2000dynamic}.
\begin{itemize}
    \item Centralized - a single node in the network is responsible for all load distribution
    \item Distributed - all nodes ale equal
\end{itemize}
While in centralized scheme are decisions made in one master workstation,
in distributed scheme, the load balancing algorithm runs on all nodes
and each node balances itself.
Each of this approach has its own ups and downs,
centralized scheme can be potential performance bottleneck
since it relies on one system node,
on the other hand distributed scheme has communication overhead,
because it requires broadcast communication between all algorithm instances.

The main advantage of dynamic load balancing is that it allows changing execution node in runtime.
For that reason it is possible to change hardware characteristics according to the job execution phase.
For example execute initial phase of optimization algorithm on machine with powerful single core performance
and then move the job to the machine with multiple, less powerful, cores to let it run in parallel.\todo{This sentence sounds weird}
Also as a result of runtime scheduling,
dynamic load balancing algorithms tend to provide a significant improvements in performance over static algorithms.
However, this comes at the additional cost of collecting and maintaining load information\cite{malik2000dynamic}.
For that reason dynamic load balancing suites better for long running tasks, which can be managed and distributed better, than for fast queries.

\subsubsection{Dynamic load balancing strategies}
There are three major parameters which usually define the strategy a specific load balancing algorithm will employ.
These three parameters answer three important questions\cite{malik2000dynamic}:
\begin{enumerate}
    \item Who makes the load balancing decision?
    \item What information is used to make the load balancing decision?
    \item Where the load balancing decision is made?
\end{enumerate}

Question number $1$ is answered based on whether a \textbf{sender-initiated} or \textbf{receiver-initiated} policy is employed.
In \textit{sender-initiated} policies, congested nodes attempt to move work to lightly-loaded nodes.
In \textit{receiver-initiated} policies, lightly-loaded nodes look for heavily-loaded nodes from which work may be received\cite{malik2000dynamic}.

\smallskip
Question `What information is used to make the load balancing decision?` is answered by following policies - \textbf{global} and \textbf{local}.
When algorithm uses \textit{global} policy, the load balancer uses the performance profiles of all execution nodes connected to the network.
When using \textit{local} policy, only local
\footnote{Workstations are usually divided into groups, in this context \textit{local} means in the same group of workstations}
nodes are taken in account while creating performance profile of the system.

\smallskip
The last parameter - `where the load balancing decision is made` - is answered by used control form,
as mentioned previously,
dynamic load balancing algorithms are divided into two groups based on their control form - \textbf{centralized} and \textbf{distributed}.

\medskip
\noindent I would like to present two general dynamic load balancing algorithms - \textit{Central Queue Algorithm} and \textit{Local Queue Algorithm}.

\subsubsection{Central Queue Algorithm}
Central queue algorithm is based on centralized receiver-initiated load balancing strategy.
It uses a cyclic FIFO queue on the main host to store new activities\footnote{Activities - jobs to be executed, in our case optimization job}
and unfulfilled requests.
New activity request is inserted into queue and here it is stored until some execution node picks it up.

Whenever a request for an activity (which is send by executing node in the case when its load has fallen bellow specified threshold)
is received by the queue manager\footnote{Queue manager - central server which manages queue},
it removes the first activity from the queue and sends it to the requester.
If the queue is empty, the request is buffered, until a new activity is available.
If a new activity arrives at the queue manager while there are unanswered requests in the queue,
the first such request is removed from the queue and the new activity is assigned to it.

When a execution node load falls under the threshold,
the local load manager sends a request for a new activity to the central load manager (which manages the central system queue).
The central load manager answers the request immediately if a ready activity is found in the queue,
or queues the request until a new activity arrives\cite{sharma2008performance}.


\subsubsection{Local Queue Algorithm}
Local queue algorithms uses distributed receiver-initiated strategy.

Its main feature is, that it supports dynamic process migration.\todo{Weird sentence}
This algorithm in the first step uses static allocation of all new processes - all processes are allocated to under loaded hosts.
In the second step the process migration is initiated by a host when its load falls under predefined threshold\footnote{
This threshold can be defined by the user and it is an input for the algorithm}.
In such case, the execution node attempts to get several processes from remote hosts.
It randomly sends requests with the number of local ready processes to remote load managers.
When a load manager receives such a request, it compares the local number of ready processes with the received number.
If the former is greater than the latter, then some of the running processes are transferred to the requester
and an affirmative confirmation with the number of processes transferred is returned.\cite{sharma2008performance}

Local queue algorithm is distributed load balancing algorithm where each execution node requests a new activity when it is under loaded.
The main advantage of using such algorithm is the fact, that there is no central point,
where all requests are managed and distributed to another segments of system.
For that reason is this particular algorithm copes and performs well under an increased or expanding workload.

\subsection{Static vs.
Dynamic scheduling}\label{subsec:static-vs.-dynamic-scheduling}
Comparison between static and dynamic scheduling\todo{There must be something about it}.
For our case dynamic is definitely better and more suitable because\ldots

\subsection{Load Balancing for Optimization Algorithms}\label{subsec:load-balancing-for-optimization-algorithms}
In general, load balancing algorithms don't use information about what exactly is being executed on the execution nodes.
This is because they are working mainly on the network layer and thus don't need that information.
Also, they are mainly designed to be generic - to be used with any system and to be suitable for every environment.
From the load balancer point of view, everything behind load balancing layer of the system is a black box.

Because there is no knowledge about the algorithms operating on the execution nodes,
load balancing algorithm can not make fully informed decision about the job execution.
However, this paper focus on the load balancing and execution scheduling of optimization algorithms,
therefore,
unlike generic load balancing solutions,
proposed load balancer \textbf{have} the information about execution algorithms on the host machine
and thus load balancing decision are more informed.
More informed load balancing decisions could potentially lead to better performance and costs reduction as well as greater capacity of whole system.

Since load balancer is aware of algorithms running on the hosts,
it can take in account a lot of execution criteria which can be specified (such as execution time)
or at least estimated (how much memory will be needed according to the domain size) in advance to make even more informed balancing decision
when scheduling job execution.
This is also the main difference between the generally used and existing load balancing software and a solution proposed in this paper.
























\newpage

%%
%% Author: lukas
%% 03.01.2019
%%

\section{Optimization Algorithms}\label{sec:optimization-algorithms}

This work does not contain any own algorithm implementation for generic optimization problems,
instead I would like to use pre-prepared and already implemented optimization solver.
First we must specify which kind of approach we would like to choose.
We have many options, how to represent and then solve scheduling problem such as
\begin{itemize}
    \item Linear programming - mixed integers programming
    \item Constraint programming - heuristics algorithms
\end{itemize}

\subsection{Linear Programming}\label{subsec:linear-programming}
Some general information about linear programming goes here %TODO add here some info

\subsubsection{Advantages of linear programming approach}
%TODO

\subsubsection{Disadvantages of linear programming approach}
%TODO

\subsubsection{Existing solutions}\label{subsec:existing-solutions-lin}
GLPK
Google Optimization Kit

\subsection{Heuristic algorithms}\label{subsec:heuristic-algorithms}
General information about heuristic %TODO add heuristic

\subsubsection{Advantages of heuristic algorithms}
The main advantage of heuristic algorithms is that they offer a quick solution for problem they are solving.

\subsubsection{Disadvantages of heuristic algorithms}
The main downside of HA is the fact, that they can't guarantee that found solution is the optimal one.

\subsubsection{Existing solutions}\label{subsec:existing-solutions-heur}
I would like to mention two implementations of heuristics algorithms - OptaPlanner and TASP.

\subsubsection{OptaPlanner}
OptaPlanner is an open source generic heuristics based constraint solver.
It is optimized to
%TODO more fom https://www.optaplanner.org

\subsubsection{TASP}
\textit{Task and Asset Scheduling Platform}  is a lightweight framework developed by Blindspot Solutions designed to solve a large
variety of optimization and scheduling problems from the area of logistics, workforce management, manufacturing, planning and others.
It contains a modular, efficient planning engine utilizing latest optimization algorithms.
TASP is delivered as a software library to be used through its API in applications which require powerful scheduling capabilities.\\
It is written in Kotlin and runs on JVM.

\subsection{Selected algorithm}\label{subsec:selected-algorithm}
I decided to use one linear solver and one heuristic algorithm to test load balancing server.
This will provide us heterogeneous environment for distinguish optimization tasks as well as different demands on performance.\\
While choosing suitable solvers I was looking mainly at possibility running on JVM and their API as well as at their suitability for my paper.
For final testing I selected \textbf{GLPK} as linear solver, mainly because it is widely used linear programming kit
and because of it's convenient Java interface.\\
As a representative of heuristics algorithms I selected \textbf{TASP} because of it's great scalability, Kotlin interface
and because I have already worked with it and I'm familiar with multiple TASP implementations.
%TODO do I have to mention that I'm working for Blindspot?




\chapter{Approach}\label{ch:approach}
Presented problem can be solved using many possible approaches,
I have decided to use mathematic optimization for running algorithms values predictions
and heuristic algorithm for load balancing decisions.

\section{Algorithm value prediction}\label{sec:algorithm-value-prediction}

Used library - https://github.com/finmath/finmath-lib


\section{Load balancing decisions}\label{sec:load-balancing-decisions}

To make informed load balancing decisions while scheduling multiple optimization algorithms,
the application uses dynamic scheduling with centralized node running load balancing algorithm.

The application also takes advantage of knowing the exact maximal time of execution thanks to the input parameter $D^{j}$ 
defined in section \ref{subsubsec:formal-input}.
Also,
because of this parameter, 
it is possible to create scheduling decisions for much larger time horizon, 
because algorithm is aware of the future workload.

The definition formalized in section \ref{sec:formal-definition} implies 
that it is possible to use integer mixed integer linear programming solver, 
because the problem itself is defined as integer linear programming problem.
That is indeed possible, 
but after careful consideration I rather decided to use heuristic approach.
The main reason for selecting this type of optimization algorithm is,
that it provides suitable results during the whole runtime.
This could be very handy when the time for load balancing decisions is tight
and in such case, the mixed integer linear programming solver would not have enough time to provide suitable solution,
because the one, it provided, would not be optimal at all.

The optimization solver, which was chosen for the application,
was previously mentioned (\ref{subsubsec:heuristic-algs-optaplanner}) heuristics optimization engine OptaPlanner.
Unlike TASP (\ref{subsubsec:tasp}), the OptaPlanner is open sourced.
The implementation based on OptaPlanner is described in section \ref{sec:load-balancing-optaplanner}.

\section{Complete application algorithm}\label{sec:olb-algorithm}

The load balancing algorithm phases are divided into scheduling rounds.
Each scheduling round produces new execution plan.
The execution plan is created for specified scheduling horizon,
this horizon is the period of time,
which is limited by the time $t+1$ and the $t_{max}$, 
which defines the end of the execution plan
and where the $t$ is the time of an execution.
In other words,
each scheduling round, executed in time $t$,
produces the scheduling plan with scheduling horizon from $t+1$ to $t_{max}$.

The scheduling horizon can be configured in the scheduling properties described in chapter \ref{ch:implementation}.
The length of scheduling window can be configured as well. 

\noindent
\begin{minipage}{\textwidth}
	The following steps describes the way, how the load balancer works.
	\begin{enumerate}
		\item \label{item:initial-scheduling-step} Poll jobs from the jobs queue.
		\begin{itemize}
			\item The incoming jobs from API are stored in the queue, 
			until the scheduling algorithm picks them up.
			That happens when the new scheduling round is engaged.
		\end{itemize}
	
		\item Analyze and filter jobs, that are not relevant for the current scheduling round.
		\begin{itemize}
			\item Filtered out are the jobs,
			 whose execution time exceeded the maximal possible execution time $D^{j}$
			 or whose cost has exceeded the maximal cost $P^{j}$.
		\end{itemize}
	
		\item Convert received jobs into the inner data representation.
		\begin{itemize}
			\item The scheduling core works with different data objects,
			than the load balancer API.
			For that reason,
			the data must be converted from the data transfer objects structure
			to the OptaPlanner domain representation.
		\end{itemize}
	
		\item Create predictions based on the solution value of the jobs 
		and on the history of the load balancing decisions.
		\begin{itemize}
			\item The process of making the predictions is described in section \ref{sec:algorithm-value-prediction}.
		\end{itemize}
	
		\item Produce execution plan using OptaPlanner scheduling.
		\begin{itemize}
			\item The scheduling is limited to the amount of time defined as scheduling window.
			When the scheduling reaches the scheduling window (for example 60 seconds),
			it is stopped.
		\end{itemize}
	
		\item Convert created plan into time schedule and send them back to the client.
		\begin{itemize}
			\item Transform internal domain representation into DTO layer and send them back to the client.
		\end{itemize}
	
		\item Engage the next scheduling round and go to the step \ref{item:initial-scheduling-step}.
	\end{enumerate}	
\end{minipage}
\bigskip

\chapter{Implementation}\label{ch:used-technologies}
In this chapter I would like to present the technologies that were used while implementing the previously described system.
During the development,
base package of the application was named \inlinecode{OLB},
which is an acronym for \textbf{O}ptimization \textbf{L}oad \textbf{B}alancer.
In the following pages, 
the developed application is called this way.

\section{Architecture}
Something about whole architecture of the application, microservices and stuff
also add some image of architecture
\subsubsection{Microservices}
pros/cons of microservices architecture in overall and my implementation

\subsubsection{Design}
packages structure, some figures with architecture visualisation



\section{Development stack}\label{sec:development-stack}
Definition of development stack
Add that target environment is JVM 11 but it has backwards compatibility to Java 8.

\subsection{Programming Language}\label{subsec:programming-language}
The OLB is not bound to the single technology, which could limit the development stack,  
and for that reason,
I had a free choice while making the decision about programming language used for the OLB implementation.

OLB is programmed in the next generation programming language \textbf{Kotlin}.
This cross-platform, statically typed, general-purpose programming language is developed by JetBrains\todo{add citations}.
Kotlin is 100\% interoperable with Java because it uses JVM\footnote{Java Virtual Machine - runtime environment for Java byte code} 
as its runtime and it is compiled to the Java Bytecode.
Apart from Java Bytecode, it can be also compiled to JavaScript or native code.\todo{citation needed}
The main advantage of Kotlin is its strong and aggressive type inference,
meaning that for the most of time,
it is not necessary to specify used data type since Kotlin compiler is able to infer it from the context.\todo{citation needed}
It results to concise language syntax and therefore to the faster development in general.

Another great advantage is Kotlin's \textit{null safety}. 
Kotlin compiler distinguishes between non-null types and nullable types 
and enforces \textit{null checks}\footnote{Check whether the object being used has not null value} when the object has nullable data type.
This feature effectively leads to less problems in code (also called \textit{bugs})
and drastically reduces \textit{Null Pointer Exceptions}\footnote{Exception raised when code access reference that has null value}
during the runtime.

\todo{maybe add something about functional approach}

\subsection{Build environment}
As a build automation system I used Gradle, 
which is high performance choice mainly because its incrementally build system,
that works by tracking input and output of tasks, 
including files changes tracking, and only running tasks, that are necessary. 
Also, it processes only these files, that were changed between tasks execution. 
Another reason I choose Gradle was, that it is preferred build system for Kotlin.

To build the application using only Gradle,
it is necessary to have installed at least JVM 8.
There are pre-prepared Gradle wrapper (\textit{gradlew}) scripts,
that are able to build the application without having the Gradle installed on the local machine.

However, the preferred approach to build the application is to use Docker
and build the application to the Docker image, 
which can be then run inside the Docker container.

\subsubsection{Docker build environment}
To keep build clean and reusable on almost every operating system and 
machine setup I decided to use \textbf{multistage Docker build} \todo{add link to docker label}
which uses different base docker images for the build and for the run phase.
Since OLB targets JVM 11 environment and uses Gradle as its build system,
\textit{gradle:5.4.0-jdk11-slim} is used as base image for build stage.
This image contains all necessary Gradle build tools while having smaller size than common Gradle Docker image.
Even smaller (in terms of size) are \textit{alpine} based docker images. 
Alpine is smallest possible Linux core, 
which is widely used in wide range of Docker base images.
Alpine is focused on the smallest possible size of the image, 
while having all necessary tools build in.
Unfortunately, there were (at the time of development) no official JVM 11 alpine images
since there is no official stable OpenJDK\footnote{Open-source implementation of the Java Platform, Standard Edition} 
11 build for Alpine Linux.

\subsection{Runtime environment}
There are multiple ways, how to start and run the application on the local machine.
\begin{itemize}
	\item As a container inside Docker system - \textbf{preferred}
	\item Locally on JVM 11
	\item Locally on the older JVM (but at least JVM 8)
\end{itemize}

The preferred runtime environment is Docker system, 
where application image runs inside the created docker container,
this is described in the next subsection.\todo{add ref to label}

Although this is preferred execution approach,
there are few other approaches,
how to start and run the application.

It is possible to start the application locally (without Docker environment) by having JVM installed directly on the machine.
Published application setup targets JVM 11, 
therefore for the successful application execution there must be present latest version of JVM 11.

However,
it is also possible to run the application on the previous versions of the JVM up to JVM 8.
Using this way of application execution means,
that it must be build directly under local JVM 8 using the Gradle build system.

\subsubsection{Docker runtime environment}\label{subsubsec:docker-runtime-env}
The build application files are copied from the Docker build stage to the Docker runtime stage.
As the runtime base image in multistage build was used \textit{openjdk:11-jre-slim} image,
because it is official OpenJDK 11 Docker image and therefore it is declared as stable.

Because there was used \textit{gradle application plugin} while building the application, 
startup scripts were generated by the Gradle.
These scripts are then used to start the application itself inside the Docker container.

When starting the whole application, 
multiple services must be started up.
Therefore, because of the containerized environment,
where containers can not access each other,
multiple containers must be started and virtual network connecting them together must be created.
This process can be automated using Docker Compose.

\subsubsection{Docker Compose}
Docker Compose is an application for defining, running and managing multi-container Docker applications.
It automatically creates Docker networks as well as Docker volumes.\todo{Maybe add more info how it works}
With writing down the definition of multiple Docker applications to the one Docker Compose configuration file,
it is possible to create powerful microservices architecture, 
which can be build or started using single command.

Thanks to the created Docker networks,
containers can communicate with each other using Docker Compose service names,
therefore they do not need to know specific IP address they are running on.

Docker Compose is used in the implementation of OLB since it is designed with microservices architecture in mind.
There are two services - Scheduling server and Scheduling client.
Scheduling server provides ability to schedule process execution on the various computers
and contains all core algorithms.
Scheduling client is an example application which uses ability of scheduling server. 
There are implemented various simulations,
which are being executed by scheduling client.  

\subsection{Framework}
KTor something

\subsubsection{Route discovery library}
My developed library for route discovery for KTro

\subsection{Build and deployment}\label{subsec:build-and-deployment}
Info how to build everything and how to deploy it to the AWS for example.


\section{Algorithm value prediction}\label{sec:algorithm-value-prediction}

Used library - https://github.com/finmath/finmath-lib


\section{Load balancing decisions}\label{sec:load-balancing-decisions}

To make informed load balancing decisions while scheduling multiple optimization algorithms,
the application uses dynamic scheduling with centralized node running load balancing algorithm.

The application also takes advantage of knowing the exact maximal time of execution thanks to the input parameter $D^{j}$ 
defined in section \ref{subsubsec:formal-input}.
Also,
because of this parameter, 
it is possible to create scheduling decisions for much larger time horizon, 
because algorithm is aware of the future workload.

The definition formalized in section \ref{sec:formal-definition} implies 
that it is possible to use integer mixed integer linear programming solver, 
because the problem itself is defined as integer linear programming problem.
That is indeed possible, 
but after careful consideration I rather decided to use heuristic approach.
The main reason for selecting this type of optimization algorithm is,
that it provides suitable results during the whole runtime.
This could be very handy when the time for load balancing decisions is tight
and in such case, the mixed integer linear programming solver would not have enough time to provide suitable solution,
because the one, it provided, would not be optimal at all.

The optimization solver, which was chosen for the application,
was previously mentioned (\ref{subsubsec:heuristic-algs-optaplanner}) heuristics optimization engine OptaPlanner.
Unlike TASP (\ref{subsubsec:tasp}), the OptaPlanner is open sourced.
The implementation based on OptaPlanner is described in section \ref{sec:load-balancing-optaplanner}.

\section{Complete application algorithm}\label{sec:olb-algorithm}

The load balancing algorithm phases are divided into scheduling rounds.
Each scheduling round produces new execution plan.
The execution plan is created for specified scheduling horizon,
this horizon is the period of time,
which is limited by the time $t+1$ and the $t_{max}$, 
which defines the end of the execution plan
and where the $t$ is the time of an execution.
In other words,
each scheduling round, executed in time $t$,
produces the scheduling plan with scheduling horizon from $t+1$ to $t_{max}$.

The scheduling horizon can be configured in the scheduling properties described in chapter \ref{ch:implementation}.
The length of scheduling window can be configured as well. 

\noindent
\begin{minipage}{\textwidth}
	The following steps describes the way, how the load balancer works.
	\begin{enumerate}
		\item \label{item:initial-scheduling-step} Poll jobs from the jobs queue.
		\begin{itemize}
			\item The incoming jobs from API are stored in the queue, 
			until the scheduling algorithm picks them up.
			That happens when the new scheduling round is engaged.
		\end{itemize}
	
		\item Analyze and filter jobs, that are not relevant for the current scheduling round.
		\begin{itemize}
			\item Filtered out are the jobs,
			 whose execution time exceeded the maximal possible execution time $D^{j}$
			 or whose cost has exceeded the maximal cost $P^{j}$.
		\end{itemize}
	
		\item Convert received jobs into the inner data representation.
		\begin{itemize}
			\item The scheduling core works with different data objects,
			than the load balancer API.
			For that reason,
			the data must be converted from the data transfer objects structure
			to the OptaPlanner domain representation.
		\end{itemize}
	
		\item Create predictions based on the solution value of the jobs 
		and on the history of the load balancing decisions.
		\begin{itemize}
			\item The process of making the predictions is described in section \ref{sec:algorithm-value-prediction}.
		\end{itemize}
	
		\item Produce execution plan using OptaPlanner scheduling.
		\begin{itemize}
			\item The scheduling is limited to the amount of time defined as scheduling window.
			When the scheduling reaches the scheduling window (for example 60 seconds),
			it is stopped.
		\end{itemize}
	
		\item Convert created plan into time schedule and send them back to the client.
		\begin{itemize}
			\item Transform internal domain representation into DTO layer and send them back to the client.
		\end{itemize}
	
		\item Engage the next scheduling round and go to the step \ref{item:initial-scheduling-step}.
	\end{enumerate}	
\end{minipage}
\bigskip

\chapter{Experiments}\label{c:experiments}

some experiments go here

\section{Optimization algorithms data}\label{sec:optimization-algorithms-data}
For the proper testing environment,
the real runtime data of the optimization algorithm were needed.
We decided to use TASP (mentioned in subsection \ref{subsubsec:tasp}) as heuristic algorithm,
which execution could be potentially scheduled by the instance of the optimization load balancer.
We did not implement a new TASP instance,
instead,
we used examples from Stochastic Dynamic Vehicle Routing Problem master's thesis by Petr Eichler\cite{Eichler:Petr:2003}.

These instances solve the real-life vehicle routing problem 
and mainly for that reason are ideal for the testing purposes.
We slightly modified the code from the thesis for observation purposes
and added the time measuring functionality,
which tracked the time between the algorithm's iterations and the current solution value of the job in each iteration.

In overall,
we executed and measured 56 different instances of TASP.
The observations can be found in \inlinedata{jobs-data/input} folder inside the implementation project
and they are being used in the simulations,
where the simulation module randomly selects one file with runtime data for each job 
and assigns it to the job, that is being scheduled.
The data are then effectively used as input data for the load balancer.

\section{Simulations}
Simulation module, particular simulation scenarios, scenarios execution

\chapter{Conclusion}\label{ch:conclusion}
This thesis focused mainly on the core of the proposed load balancing system
and outlined the way,
how to make more informative decisions while scheduling optimization algorithms execution in heterogenous system.

The experiments showed,
that this kind of tasks scheduling is possible to use and sustainable in larger systems.

aaaaaaaaaaa\todo{here}

\section{Future work}\label{sec:future-work}
Although the core part of the scheduling server for optimization algorithm was developed in this thesis,
there is lot of work being left.
The main missing part is an execution module, 
which would transform the created plan into physical actions performed in the infrastructure 
like running, stopping and moving the jobs between the physical execution nodes.

\subsubsection{Infrastructure development}
When the previously mentioned execution module is implemented,
the solution can be deployed into real life environment and properly tested.
Proposition is to make optimization algorithms running as Docker containers 
and the execution module would operate with Docker machines,
which would be running on the execution nodes. 
In this way, 
the assigned resources would be very easily changed on one execution node.
Migration to the next execution node would not be problem as well, 
since containers could be wrapped and transported through the network.

\subsubsection{Documentation and unit testing}
Another thing, that must be improved, is documentation as well as unit testing of the application.
Unfortunately, the current lack of code documentation could discourage from future development of the system.
As for the unit tests, 
only crucial parts of application such as extension functions and prediction module are at least partially unit tested.
The goal is to have documented and tested whole core module,
which is responsible for the scheduling itself and public APIs of server modules. 

\subsubsection{Routes discovery library}
Routes discovery library was very handy during the development of application's server parts 
and I believe that this way of routes registration in Ktor would suit to many developers as well.

Therefore I would like to refactor it from the base project 
and create open source project which will ensure future library development.


\bibliography{literature}
\end{document}