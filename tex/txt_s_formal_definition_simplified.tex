%%
%% Author: lukas
%% 12.02.2019
%%

\section{Simplified formal definition}\label{sec:formal-definition-simplified}
This more general and simplified formal definition covers main parts of the optimization problem 
and does not use excessive definition of all possible variables.

It treats resources as homogenous set of all possible resource combination,
therefore resources set contains all possible combination of CPU/RAM configuration that is available.
Thanks to this simplification, resources assignment is can be binary,
where one represents assignment and zero represents no assignment.

It also omits data variable as well as underlying algorithm and resources provider index, 
in favor of better readability.


\subsection{Variables Definition}\label{subsec:variables-definition}
Following indexes, inputs and variables are used in the optimization criteria.

\subsubsection{Indexes}
\begin{itemize}
	\item $j$ - index used to identify something related to the execution job,
	      in real world this is most likely job id, located in the right upper corner - $x^{j}$, 
	      set of all jobs in the system is represented by $J$
	\item $r$ - index used to identify resources, written in the left upper corner - ${}^{r}x$, set of resources is represented by $R$
	\item $t$ - right bottom index represents time - $x_t$
\end{itemize}

\subsubsection{Input}
Input which is specified before executing optimization job by the user outside of the system.
When new execution job is requested - this is done by the user, or client application,
following data should be provided.
\begin{itemize}
	\item $D^{j}$ - maximal duration of the job execution which cannot be exceeded
	\item $P^{j}$ - maximal used resources cost per job, or in other words highest possible price paid for the job execution which cannot be exceeded
\end{itemize} 
There are also constants defined before load balancer start up.
\begin{itemize}
	\item ${}^{r}c$ - cost of the using particular resources per one time unit
\end{itemize}
Each of the previously mentioned variable must be non-negative.

\subsubsection{Program Output}
Apart from result of the underlying optimization algorithm,
following data are returned to user after successful job execution.

\begin{itemize}
	\item $T^{j}$ - time taken, duration of the actual job execution
	\item $C^{j}$ - resource costs, how much job execution cost
\end{itemize}

\subsubsection{Variables}

\begin{itemize}
	\item $v_{t}^{j}$ - value\todo{better example would be fine} (i.e.\ cost of the scheduled plan) of the job $j$ at the time $t$
	      Value is greater than zero and it is non-increasing during the time. 
	      \begin{align*}
	      	\forall t, j: v_{t}^{j} \geq v_{t+1}^{j} > 0 
	      \end{align*}
	      It is non-increasing because optimization algorithms return always best found solution, 
	      so when worse solution, than currently best one, is found,
	      returned is still the best solution found.
	\item $^{r}x_{t}^{j}$ - represents assignment of the resources $r$ at time $t$ to job $j$
	      \begin{align*}
	      	^{r}x_{t}^{j} = \{0, 1\} 
	      \end{align*}
	      Each $x$ is either $1$ = indexed resources are assigned to the job at given time or $0$ = given combination does not have assignment.
	      We assume that each job has only one such assignment at one time,
	      which effectively means that this job is executed on the single computation node.
	      This is defined by following constraint:
	      \begin{align*}
	      	\forall j, t:\; \sum_{r \in R} {}^{r}x_{t}^{j} = \{0, 1\} 
	      \end{align*}
	\item $^{r}\Delta_{t}^{j}$ - enhancement of the value $v$ with resources $r$ on the job $j$ per time $t$.
	      \begin{align*}
	      	^{r}\Delta_{t}^{j} = {}^{r}| v_{t}^{j} - v_{t-1}^{j}| 
	      \end{align*}
	      It is improvement of the solution value $v$ which can be achieved by using resources $r$ at time $t$.
	      This value is always non-negative since optimization algorithm always stores best found solution,
	      and therefore $\forall j, t, r:\; {}^{r}\Delta_{t}^{j} \geq 0$. 
	\item $S_{t}^{j}$ - reward for improving solution value in time $t$ per job $j$.
	      Accumulation of enhancements $^{r}\Delta_{t}^{j}$ through all resources $r$ and time units $t$.
	      \begin{align*}
	      	S_{t}^{j} = \sum_{i = 0}^{t}\:\sum_{r \in R}\: {}^{r}\Delta_{i}^{j} 
	      \end{align*}
	\item $C_{t}^{j}$ - defines how much execution of job $j$ cost from the beginning of the execution until time $t$.
	      Sum of all allocated resources for their time for the particular job.
	      \begin{align*}
	      	C_{t}^{j} = \sum_{i = 0}^{t}\:\sum_{r \in R}\: {}^{r}c \cdot {}^{r}x_{t}^{j} 
	      \end{align*}
	      Because $C_{t}^{j}$ is defined as sum and ${}^{r}c$ is non-negative,
	      it is true that $\forall j, t:\; C_{t+1}^{j} \geq C_{t}^{j}$.
	      Input of the program specifies maximal cost paid for job execution as a $P^j$, 
	      therefore it must be enforced by the system that this cost will not be exceeded.
	      This constraint can be defined as follows.
	      \begin{align*}
	      	\forall t, j:\; P^{j} \leq C_{t}^{j} \implies \sum_{i = t+1}^{\infty} \, \sum_{r \in R} {}^{r}x_{i}^{j} = 0 
	      \end{align*}
	      Which effectively means that when cost of job execution $C_{t}^{j}$ has reached maximal defined cost $P^{j}$,
	      no resources can be assigned to this job.
	\item $t$ - time, it is not only index but also variable, there are also constraints regarding time -
	      since client application can specify deadline to job $D^{j}$,
	      there must be additional constraint for job execution in a matter of resources assignment.
	      \begin{align*}
	      	\forall t, j:\; D^{j} \leq t \implies \sum_{i = t+1}^{\infty} \, \sum_{r \in R} {}^{r}x_{i}^{j} = 0 
	      \end{align*}
	      When maximal time is exceeded, no additional resources can be assigned to the job execution, 
	      which could be defined by following constraint.
\end{itemize}

\subsection{Resources reconfiguration}\label{subsec:resource-config}
System should be capable of changing resources assignment per job in the runtime.
This will help to distribute performance according to the current nodes load across whole network
and allow to 
Unfortunately, it is not always possible to reconfigure resources assignment while scheduling is being performed.
Therefore there must be at least one time unit, between different resources assignment,
where no resources are assigned to the job.
In other words, if resources reconfiguration is being done in time $t$ then $\sum_{r \in R} {}^{r}x_{t}^{j} = 0$.

This constraint can be defined in pseudocode,
where $m, n \in R$, $m$ are assigned resources at the time $t$,
this be written as  ${}^{m}x_{t}^{j} = 1$,
and $n$ are resources that should be assigned to the job in the time $t+1$.

\medskip

\begin{samepage}
	\begin{algorithmic}
		\IF {$ {}^{m}x_{t}^{j} = 1$} 
		\IF {$m = n$}
		\STATE ${}^{n}x_{t+1}^{j} = 1$
		\ELSE
		\STATE $\sum_{r \in R} {}^{r}x_{t+1}^{j} = 0$
		\ENDIF 
		\ENDIF 
	\end{algorithmic}        
\end{samepage}

\medskip

This constraint can defined as a mathematical function.\todo{Add mathematical definition of this constraint, is it even possible?}

\subsection{Optimization criteria}\label{subsec:optimization-criteria}
The main goal of the system is to minimize outcome value of the underlining optimization algorithm
and at the same time to minimize cost of used resources.
We can optimize single job or sum of outcomes from all jobs in the system at once.
First approach provides possibility to control and optimize outcome of a particular job,
which is an advantage for single client (owner o the job),
but it does not necessarily means that it is optimal for the whole system and vice versa.
Optimization criteria for the single job at particular time $t$ is then
maximization of weighted difference between the value enhancement reward 
and cost paid for the enhancement,
which can be described by following equation.
\begin{align*}
	\max crit_{t}^{j} = \alpha \cdot S_{r}^{j} - (1 - \alpha) \cdot C_{t}^{j} \qquad 0 \leq \alpha \leq 1 
\end{align*}
For optimization of system-wide resources and costs,
all jobs execution optimization is then defined like a
weighted sum of all rewards per jobs lowered by sum of all resources costs across the set of all jobs.
\begin{align*}
	\max crit_{t} = \alpha \sum_{j \in J} S_{r}^{j} - (1 - \alpha) \sum_{j \in J} C_{t}^{j} \qquad 0 \leq \alpha \leq 1 
\end{align*}
