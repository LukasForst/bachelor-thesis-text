\section{Complete application algorithm}\label{sec:olb-algorithm}

There are two possible ways, 
how to create execution plan with load balancing decisions.
The first should be used when no scheduled optimization algorithm runtime data are available.
Typically, when system is starting up
and no job is being currently scheduled.
The second aims to provide more informed decisions by using jobs runtime data with job-value function history.
Thanks to this provided history,
load balancing algorithm is able to take in account the prediction of future job-value function development.

Common thing for the two scheduling ways is,
that they need to receive data set, 
convert it into inner core representation
and after plan creation send the scheduled data to the client.

The final algorithm is then following, 
the two phases are described below.
\begin{algorithm}[H]
    \SetAlgoLined
    \textbf{jobsDomain} $\leftarrow$ gather jobs from waiting queue\;
    \textbf{plan} $\leftarrow$ execute initial initial plan creation on \textbf{jobsDomain}\;
	\While{job to schedule exists} {
        \textbf{waitingJobs} $\leftarrow$ gather jobs from waiting queue\;
        \textbf{jobsExecutionData} $\leftarrow$ gather runtime information about jobs runtime\;
        \textbf{plan} $\leftarrow$ enhance \textbf{plan} with \textbf{jobsExecutionData} 
        and \textbf{waitingJobs} $\cap$ \textbf{jobsDomain}\;
	}
\end{algorithm} 

\subsection{Initial plan creation}\label{subsec:initial-plan-creation}

In the initial plan creation phase, 
there are no data to begin with.
Therefore the prediction functionality is not used.
Also, 
the predictions dependents constraints in core OptaPlanner wrapper implementation are skipped. 

The following algorithm steps are used when the initial plan creation is executed. 

\begin{algorithm}[H]
	\SetAlgoLined
	\Input{jobs, scheduling properties}
	\textbf{jobsToSchedule} $\leftarrow$ convert given jobs to inner core representation\;
	\textbf{plan} $\leftarrow$ start OptaPlanner scheduling with \textbf{jobsToSchedule}\;
	convert created \textbf{plan} to output data\;
	\Output{Time schedule for each job}
\end{algorithm} 

\subsection{Plan enhancement}\label{subsec:plan-enhancement}
During the plan enhancement scheduling stage are used all available information for more informed decisions.

Following pseudocode describes behavior of the program when the plan enhancement phase is triggered.
The obsolete data are jobs,
that were already scheduled and executed,
or are due their scheduling history unschedulable (their $D^{j}$ or $P^{j}$ exceeded).

\begin{algorithm}[H]
	\SetAlgoLined
    \Input{jobs, scheduling properties, job history}
    \textbf{jobsDomain} $\leftarrow$ filter obsolete data from jobs\;
    \textbf{jobsToSchedule} $\leftarrow$ convert \textbf{jobsDomain} to inner core representation\;
    \textbf{predictions} $\leftarrow$ create predictions of job-value function based on job history data\;
	\textbf{plan} $\leftarrow$ start OptaPlanner scheduling with \textbf{jobsToSchedule} and \textbf{predictions}\;
	convert created \textbf{plan} to output data\;
	\Output{Time schedule for each job}
\end{algorithm} 
