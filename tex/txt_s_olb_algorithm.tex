\section{Complete application algorithm}\label{sec:olb-algorithm}

The load balancing algorithm phases are divided into scheduling rounds.
Each scheduling round produces new execution plan.
The execution plan is created for specified scheduling horizon,
this horizon is the period of time,
which is limited by the time $t+1$ and the $t_{max}$, 
which defines the end of the execution plan
and where the $t$ is the time of an execution.
In other words,
each scheduling round, executed in time $t$,
produces the scheduling plan with scheduling horizon from $t+1$ to $t_{max}$.

The scheduling horizon can be configured in the scheduling properties described in chapter \ref{ch:implementation}.
The length of scheduling window can be configured as well. 

\noindent
\begin{minipage}{\textwidth}
	The following steps describes the way, how the load balancer works.
	\begin{enumerate}
		\item \label{item:initial-scheduling-step} Poll jobs from the jobs queue.
		\begin{itemize}
			\item The incoming jobs from API are stored in the queue, 
			until the scheduling algorithm picks them up.
			That happens when the new scheduling round is engaged.
		\end{itemize}
	
		\item Analyze and filter jobs, that are not relevant for the current scheduling round.
		\begin{itemize}
			\item Filtered out are the jobs,
			 whose execution time exceeded the maximal possible execution time $D^{j}$
			 or whose cost has exceeded the maximal cost $P^{j}$.
		\end{itemize}
	
		\item Convert received jobs into the inner data representation.
		\begin{itemize}
			\item The scheduling core works with different data objects,
			than the load balancer API.
			For that reason,
			the data must be converted from the data transfer objects structure
			to the OptaPlanner domain representation.
		\end{itemize}
	
		\item Create predictions based on the solution value of the jobs 
		and on the history of the load balancing decisions.
		\begin{itemize}
			\item The process of making the predictions is described in section \ref{sec:algorithm-value-prediction}.
		\end{itemize}
	
		\item Produce execution plan using OptaPlanner scheduling.
		\begin{itemize}
			\item The scheduling is limited to the amount of time defined as scheduling window.
			When the scheduling reaches the scheduling window (for example 60 seconds),
			it is stopped.
		\end{itemize}
	
		\item Convert created plan into time schedule and send them back to the client.
		\begin{itemize}
			\item Transform internal domain representation into DTO layer and send them back to the client.
		\end{itemize}
	
		\item Engage the next scheduling round and go to the step \ref{item:initial-scheduling-step}.
	\end{enumerate}	
\end{minipage}
\bigskip