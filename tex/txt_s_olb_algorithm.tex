\section{Complete application algorithm}\label{sec:olb-algorithm}

The load balancing algorithm phases are divided into scheduling rounds.
Each scheduling round produces new execution plan.
The execution plan is created for specified scheduling horizon,
this horizon is the period of time,
which is limited by the time $t+1$ and the $t_{max}$, 
which defines the end of the execution plan
and where the $t$ is the time of an execution.
In other words,
each scheduling round, executed in time $t$,
produces the scheduling plan with scheduling horizon from $t+1$ to $t_{max}$.
The scheduling horizon can be configured in the scheduling properties described in chapter \ref{ch:implementation}.
The length of scheduling window can be configured as well. 

The formalized algorithm is then presented in the algorithm~\ref{alg:load-balancing-alg}.
The job related properties $D,T,P$ and $C$ are the very same variables defined in the section \ref{sec:formal-definition}.
The functions mentioned in the algorithm are described in the algorithm detailed description listed bellow the algorithm itself.
The function \textit{predict} is related to the step described in the step \ref{item:predictions-scheduling-step}
and the function \textit{schedule} to the step \ref{item:scheduling-step}.

\bigskip
\begin{algorithm}[H]
	\SetAlgoLined
	\Input{$Q$ - queue with jobs to schedule}
	$Q$: jobs queue\;
	$J$: set of jobs\;
	$P$: predictions\;
	$E$: execution plan\;
	\BlankLine
	$J \leftarrow$ $Q$.poll()\;
	$J'\leftarrow$ $J$.filter($job \rightarrow job.D > job.T \wedge job.P > job.C$)\;
	$J'' \leftarrow$ convert($J'$)\;
	$P \leftarrow$ predict($J''$)\;
	$E \leftarrow$ schedule($J''$, $P$)\;
	$E' \leftarrow$ convert($E$)\;
	\Result{$Q$ is empty}
	\Output{$E'$ - execution plan}
	\caption{Load balancing algorithm}
	\label{alg:load-balancing-alg}
\end{algorithm} 
\medskip
\begin{enumerate}
	\itemsep0pt
	\item \label{item:initial-scheduling-step} Poll jobs from the jobs queue.
	\begin{itemize}
		\item The incoming jobs from API are stored in the queue, 
		until the new scheduling round is engaged.
	\end{itemize}

	\item Analyze and filter jobs, that are not relevant for the current scheduling round.
	\begin{itemize}
		\item Filtered out are the jobs,
		 whose execution time exceeded the maximal possible execution time $D^{j}$
		 or whose cost has exceeded the maximal cost $P^{j}$.
	\end{itemize}

	\item \label{item:convert-to-scheduling-step} Convert received jobs into the inner data representation.
	\begin{itemize}
		\item The data must be converted from the data transfer objects structure
		to the OptaPlanner domain representation.
	\end{itemize}

	\item \label{item:predictions-scheduling-step} Create predictions based on the solution value of the jobs 
	and on the history of the load balancing decisions.
	\begin{itemize}
		\item The process of making the predictions is described in section \ref{sec:algorithm-value-prediction}.
	\end{itemize}

	\item \label{item:scheduling-step} Produce execution plan using OptaPlanner scheduling.
	\begin{itemize}
		\item The scheduling is limited to the amount of time defined as scheduling window.
		When the scheduling reaches the scheduling window (for example 60 seconds),
		it is stopped.
	\end{itemize}

	\item \label{item:convert-back-scheduling-step} Convert created plan into time schedule and send them back to the client.

	\item Engage the next scheduling round and go to the step \ref{item:initial-scheduling-step}.
\end{enumerate}	