\section{Simulations}\label{sec:simulations}

There are two main randomized simulation scenarios inside simulation module 
(module described in \ref{subsec:simulations-architecture}).
Both simulation can be started up locally using \inlinecode{SimulationExecutor} or
using \inlinecode{docker-compose} microservices runtime environment.
Input data for tests are created randomly,
based on number of scheduled jobs and given planning horizon (how much steps ahead should scheduler count with).
The number of scheduled jobs can be easily edited directly in code.
For first simulation data and input please refer to \inlinecode{OnePlanningRoundSimulation},
for second to \inlinecode{RuntimeSimulation}.

Simulation's output (produced solution) is printed to standard output.
Data displayed in listing~\ref{lst:data-example} are an output produced by simulation with 5 jobs being scheduled at once.
\medskip
\begin{samepage}
\begin{lstlisting}[caption={Simulation data output},label={lst:data-example},language=Kotlin]
                Times:||  0| 60|120|180|240|300|360|
    ----------------- || --|---|---|---|---|---|---|
        Cost: 1 + 0.02||  2|  2|  4|  3|  3|  2|  2|
        Cost: 1 + 0.02||  2|  2|  4|  3|  3|  2|  2|
        Cost: 1 + 0.02||  2|  2|  4|  3|  3|  2|  2|
        Cost: 1 + 0.02||  3|  4|  4|  3|  3|  2|  2|
        Cost: 1 + 0.02||  4|  4|  4|  3|  3|  2|  2|

      Cost: 1.5 + 0.02||  1|  1|  1|  2|   |   |   |
      Cost: 1.5 + 0.02||  1|  1|  1|  2|   |   |   |

     Cost: 10.0 + 0.05||  0|  0|   |   |   |   |   |
     Cost: 10.0 + 0.05||  0|  0|   |   |   |   |   |

    Total plan cost: CostImpl(value=321.08)$
\end{lstlisting}
\end{samepage}
\medskip \noindent
\inlinecode{Times} axis shows time units ($t$ value described in section \ref{sec:formal-definition}) in seconds.
Second, vertical axis, visualizes resources.
Name, \inlinecode{Cost: 1 + 0.02} is name of the resources provider (explained in section \ref{subsec:formalized-definition-representation}),
and each line is one usage unit - meaning that it could be one physical core or for example percentage of shared processor.
In implementation, this value is referred as CPU core,
but in fact, it is dimensionless value expressing usage of some system.

Each cell contains either number or is empty.
Number is job ID and indicates, that this resource is allocated to job with displayed ID.
This is effectively $^{r}x_{t}^{j}$ value explained in section \ref{subsec:variables-definition}.

Total plan cost is sum of all particular costs of all jobs - their resources allocations.
Therefore this is the cost of created plan.

First simulation is designed to use only first scheduling stage described in section \ref{sec:olb-algorithm}
and create static initial plan which is then not modified.
This simulation is implemented in \inlinecode{OnePlanningRoundSimulation} class.

TODO - more information about simulations