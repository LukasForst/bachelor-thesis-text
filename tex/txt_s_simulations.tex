\section{Simulations}\label{sec:simulations}

There are two main randomized simulation scenarios inside simulation module 
(module described in \ref{subsec:simulations-architecture}).
Both simulation can be started up locally using \inlinecode{SimulationExecutor} or
using \inlinecode{docker-compose} microservices runtime environment.
Input data for tests are created randomly,
based on number of scheduled jobs and given planning horizon (how much steps ahead should scheduler count with).
The number of scheduled jobs can be easily edited directly in code.
For first simulation data and input please refer to \inlinecode{OnePlanningRoundSimulation},
for second to \inlinecode{RuntimeSimulation}.

Simulation's output (produced solution) is printed to standard output.
And has following structure.

This is an output example produced by simulation with 5 jobs scheduled at once.
\medskip
\begin{samepage}
\begin{lstlisting}[language=Kotlin]
    0 iteration:
                Times:||  0| 60|120|180|240|300|360|
    ----------------- || --|---|---|---|---|---|---|
        Cost: 1 + 0.02||  2|  2|  4|  3|  3|  2|  2|
        Cost: 1 + 0.02||  2|  2|  4|  3|  3|  2|  2|
        Cost: 1 + 0.02||  2|  2|  4|  3|  3|  2|  2|
        Cost: 1 + 0.02||  3|  4|  4|  3|  3|  2|  2|
        Cost: 1 + 0.02||  4|  4|  4|  3|  3|  2|  2|

      Cost: 1.5 + 0.02||  1|  1|  1|  2|   |   |   |
      Cost: 1.5 + 0.02||  1|  1|  1|  2|   |   |   |

     Cost: 10.0 + 0.05||  0|  0|   |   |   |   |   |
     Cost: 10.0 + 0.05||  0|  0|   |   |   |   |   |

    Total plan cost: CostImpl(value=321.08)$
\end{lstlisting}
\end{samepage}

First simulation is designed to use only first scheduling stage described in section \ref{sec:olb-algorithm}
and create static initial plan which is then not modified.
This simulation is implemented in \inlinecode{OnePlanningRoundSimulation} class.
