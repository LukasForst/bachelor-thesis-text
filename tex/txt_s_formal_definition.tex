%%
%% Author: lukas
%% 12.02.2019
%%

\section{Formal definition}\label{sec:formal-definition}
In the first place, used variables have to be introduced.\todo{this must be changed}

\subsection{Variables Definition}\label{subsec:variables-definition}
\todo{Maybe write something here}

\subsubsection{Indexes}
\begin{itemize}
	\item $j$ - index used to identify something related to the execution job,
	      in real world this is most likely job id, located in the right upper corner - $x^{j}$, 
	      set of all jobs in the system is represented by $J$
	\item $r$ - index used to identify resources, written in the left upper corner - ${}^{r}x$, set of resources is represented by $R$
	\item $t$ - right bottom index represents time - $x_t$
\end{itemize}

\subsubsection{Input}
Input which is specified before executing optimization job by user outside of the system.

\begin{itemize}
	\item $D^{j}$ - maximal duration of the job execution which cannot be exceeded
	\item $P^{j}$ - maximal used resources cost per job, or in other words highest possible price paid for the job execution which cannot be exceeded
\end{itemize}
These are values provided by user of the load balancer.
There are also constants defined before load balancer start up.
\begin{itemize}
	\item ${}^{r}c$ - cost of the using particular resources for one time unit
\end{itemize}
For each previously mentioned variable applies that its value is $\geq 0$.

\subsubsection{Program Output}
Apart from result of optimization algorithm,
following data are returned to user after successful job execution.

\begin{itemize}
	\item $V^{j}$ - solution value provided by algorithm
	\item $T^{j}$ - time taken, duration of the actual job execution
	\item $C^{j}$ - resource costs, how much job execution cost
\end{itemize}

\subsubsection{Variables}

\begin{itemize}
	\item $v_{t}^{j}$ - value\todo{better example would be fine} (i.e.\ cost of the scheduled plan) of the job $j$ at the time $t$
	      This value must be greater than zero and 
	      \begin{align*}
	      	\forall t, j: v_{t}^{j} \geq v_{t+1}^{j} > 0 
	      \end{align*}
	\item $^{r}x_{t}^{j}$ - represents assignment of the resources $r$ at time $t$ to job $j$
	      \begin{align*}
	      	^{r}x_{t}^{j} = \{0, 1\} 
	      \end{align*}
	      Each $x$ is either $1$ = indexed resources are assigned to the job at given time or $0$ = given combination does not have assignment.
	      We assume that each job has only one such assignment at one time,
	      which effectively means that this job is executed on the single computation node.
	      This is defined by following constraint:
	      \begin{align*}
	      	\forall j, t:\; \sum_{r \in R} {}^{r}x_{t}^{j} = \{0, 1\} 
	      \end{align*}
	\item $^{r}\Delta_{t}^{j}$ - enhancement of the value $v$ with resources $r$ on the job $j$ per time $t$.
	      \begin{align*}
	      	^{r}\Delta_{t}^{j} = {}^{r}| v_{t}^{j} - v_{t-1}^{j}| 
	      \end{align*}
	      It is improvement of the solution value $v$ which can be achieved by using resources $r$ at time $t$.
	      This value 
	\item $S_{t}^{j}$ - reward for improving solution value in time $t$ per job $j$.
	      Accumulation of enhancements $^{r}\Delta_{t}^{j}$ through all resources $r$ and time units $t$.
	      $R$ in following equation represents set of all available resources in the system.
	      \begin{align*}
	      	S_{t}^{j} = \sum_{i = 0}^{t}\:\sum_{r \in R}\: {}^{r}\Delta_{i}^{j} 
	      \end{align*}
	\item $C_{t}^{j}$ - defines how much execution of job $j$ cost from the beginning of the execution until time $t$.
	      Sum of all allocated resources for their time for the particular job.
	      \begin{align*}
	      	C_{t}^{j} = \sum_{i = 0}^{t}\:\sum_{r \in R}\: {}^{r}c \cdot {}^{r}x_{t}^{j} 
	      \end{align*}
\end{itemize}

\subsubsection{Constraints}
There are plenty of constraints applying to the variables.
\todo{Add comment what each of this constraint means}
\begin{align*}
	\forall t, j:\; v_{t}^{j} \geq v_{t+1}^{j} > 0                                                              \\     
	\forall t, j:\; \sum_{r \in R} {}^{r}x_{t}^{j} = \{0, 1\}                                                   \\
	\forall t, j:\; D^{j} \leq t \implies \sum_{i = t+1}^{\infty} \, \sum_{r \in R} {}^{r}x_{i}^{j} = 0         \\
	\forall t, j:\; P^{j} \leq C_{t}^{j} \implies \sum_{i = t+1}^{\infty} \, \sum_{r \in R} {}^{r}x_{i}^{j} = 0 \\
	\forall t, j, r:\; {}^{r}\Delta_{t}^{j} \geq 0                                                              \\
	\forall j, t:\; C_{t+1}^{j} \geq C_{t}^{j}                                                                  
\end{align*}

\subsubsection{Resources reconfiguration}
System should be capable of changing resources assignment per job.
This will help to the system to distribute performance according to the current nodes load across whole network.
Unfortunately, it is not always possible to reconfigure resources assignment while scheduling is being performed.
Therefore there must be at least one time unit, between different resources assignment,
where no resources are assigned to the job.
In other words, if resources reconfiguration is being done in time $t$ then $\sum_{r \in R} {}^{r}x_{t}^{j} = 0$.

This constraint can be defined in pseudocode,
where $m, n \in R$, $m$ are assigned resources at the time $t$,
this assignment can be written ass  ${}^{m}x_{t}^{j} = 1$
and $n$ are resources that should be assigned to the job in the time $t+1$ - ${}^{n}x_{t}^{j}$.

\begin{samepage}
	\begin{algorithmic}
		\IF {$ {}^{m}x_{t}^{j} = 1$} 
		\IF {$m = n$}
		\STATE ${}^{n}x_{t+1}^{j} = 1$
		\ELSE
		\STATE $\sum_{r \in R} {}^{r}x_{t+1}^{j} = 0$
		\ENDIF 
		\ENDIF 
	\end{algorithmic}        
\end{samepage}

This constraint can defined as a mathematical function.\todo{Add mathematical definition of this constraint}

\subsubsection{Optimization criteria}
The main goal of the system is to minimize outcome value of the underlining optimization algorithm
and at the same time to minimize cost per used resources.
We can optimize single job or sum of outcomes from all jobs in the system at once.
First approach provides possibility to control and optimize outcome of a particular job,
which is an advantage for single client (owner o the job),
but it does not necessarily means that it is optimal for the whole system and vice versa.
Optimization criteria for the single job at particular time $t$ is then
maximization of weighted difference between the value enhancement reward 
and cost paid for the enhancement,
which can be described by following equation.
\begin{align*}
	\max crit_{t}^{j} = \alpha \cdot S_{r}^{j} - (1 - \alpha) \cdot C_{t}^{j} \qquad 0 \leq \alpha \leq 1 
\end{align*}
For optimization of system-wide resources and costs,
all jobs execution optimization is then defined like a
weighted sum of all rewards per jobs lowered by sum of all resources costs across the set of all jobs.
\begin{align*}
	\max crit_{t} = \alpha \sum_{j \in J} S_{r}^{j} - (1 - \alpha) \sum_{j \in J} C_{t}^{j} \qquad 0 \leq \alpha \leq 1 
\end{align*}
