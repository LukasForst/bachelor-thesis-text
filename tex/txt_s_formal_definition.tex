%%
%% Author: lukas
%% 12.02.2019
%%

\section{Formal definition}\label{sec:formal-definition}
In the first place, used variables have to be introduced.\todo{this must be changed}

\subsection{Variables Definition}\label{subsec:variables-definition}
\todo{Maybe write something here}

\subsubsection{Indexes}
\begin{itemize}
    \item $j$ - index used to identify something related to the execution job, in real world this is most likely job id, located in the right upper corner - $x^{j}$
    \item $r$ - index used to identify resources, written in the left upper corner - ${}^{r}x$, whole set of resources is represented by letter $R$, therefore $r \in R$
    \item $t$ - right bottom index represents time - $x_t$
\end{itemize}

\subsubsection{Input}
Input which is specified before executing optimization job by user outside of the system.

\begin{itemize}
    \item $D^{j}$ - maximal duration of the job execution which cannot be exceeded
    \item $P^{j}$ - maximal used resources cost per job, or in other words highest possible price paid for the job execution which cannot be exceeded
\end{itemize}
These are values provided by user of the load balancer.
There are also constants defined before load balancer start up.
\begin{itemize}
    \item ${}^{r}c$ - cost of the using particular resources for one time unit
\end{itemize}
For each previously mentioned variable applies that its value is $\geq 0$.

\subsubsection{Program Output}
Apart from result of optimization algorithm,
following data are returned to user after successful job execution.

\begin{itemize}
    \item $V^{j}$ - solution value provided by algorithm
    \item $T^{j}$ - time taken, duration of the actual job execution
    \item $C^{j}$ - resource costs, how much job execution cost
\end{itemize}

\subsubsection{Variables}

\begin{itemize}
    \item $v_{t}^{j}$ - value\todo{better example would be fine} (i.e.\ cost of the scheduled plan) of the job $j$ at the time $t$
    \begin{align*}
        \forall t, j: v_{t}^{j} \geq v_{t+1}^{j} > 0
    \end{align*}
    \item $^{r}x_{t}^{j}$ - represents assignment of the resources $r$ at time $t$ to job $j$
    \begin{align*}
        ^{r}x_{t}^{j} = \{0, 1\}
    \end{align*}
    Each $x$ is either $1$ = indexed resources are assigned to the job at given time or $0$ = given combination does not have assignment.
    We assume that each job has only one such assignment at one time,
    which effectively means that this job is executed on the single computation node.
    This is defined by following constraint:
    \begin{align*}
        \forall j, t:\; \sum_{r \in R} {}^{r}x_{t}^{j} = \{0, 1\}
    \end{align*}
    \item $^{r}\Delta_{t}^{j}$ - enhancement of the value $v$ with resources $r$ on the job $j$ per time $t$.
    \begin{align*}
        ^{r}\Delta_{t}^{j} = {}^{r}| v_{t}^{j} - v_{t-1}^{j}|
    \end{align*}
    Word definition - improvement of the solution value $v$ which can be achieved by using resources $r$ at time $t$.
    \item $S_{t}^{j}$ - reward for improving solution value in time $t$ per job $j$.
    Accumulation of enhancements $^{r}\Delta_{t}^{j}$ through all resources $r$ and time units $t$.
    $R$ in following equation represents set of all available resources in the system.
    \begin{align*}
        S_{t}^{j} = \sum_{i = 0}^{t}\:\sum_{r \in R}\: {}^{r}\Delta_{i}^{j}
    \end{align*}
    \item $C_{t}^{j}$ - defines how much execution of job $j$ cost from the beginning of the execution until time $t$.
    Sum of all allocated resources for their time for the particular job.
    \begin{align*}
        C_{t}^{j} = \sum_{i = 0}^{t}\:\sum_{r \in R}\: {}^{r}c \cdot {}^{r}x_{t}^{j}
    \end{align*}
\end{itemize}

\subsubsection{Constraints}
There are plenty of constraints applying to the variables.

\begin{align*}
    \forall t, j: v_{t}^{j} \geq v_{t+1}^{j} > 0
\end{align*}









































