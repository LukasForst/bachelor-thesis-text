\chapter{Experiments}\label{ch:experiments}

In this chapter, 
we would like to present,
how we tested the load balancer.
The simulations were designed to reflect the real-life situations 
and to simulate the most common usage of the load balancer.


\section{Simulations implementation}\label{sec:simulations-implementations}
To test the load balancer,
there are two modules, 
that contain simulation code used for testing -
\inlinecode{simulation} and \inlinecode{remote-scheduler} module.
The simulation engine and all simulation use cases are located in the simulation module.
The second mentioned module contains server, 
which runs simulations on the remote API. 
This is especially useful when testing whole scheduling environment
and it is as closest as possible to the real-life environment,
which should be based on the microservices architecture.

Both simulation can be started up locally using \inlinecode{SimulationExecutor} or
using \inlinecode{docker-compose} inside the microservices runtime environment.
Input data for the tests are created randomly,
based on number of scheduled jobs and given planning horizon (how much steps ahead should scheduler count with).
The number of scheduled jobs can be easily edited directly in code.
For the simulation data and the input configuration please refer to \inlinecode{OnePlanningRoundSimulation} 
and \inlinecode{RuntimeSimulation} classes.

\section{Optimization algorithms data}\label{sec:optimization-algorithms-data}
For the proper testing environment,
the real runtime data of the optimization algorithm were needed.
We decided to use TASP (mentioned in subsection \ref{subsubsec:tasp}) as heuristic algorithm,
which execution could be potentially scheduled by the instance of the optimization load balancer.
We did not implement a new TASP instance,
instead,
we used examples from Stochastic Dynamic Vehicle Routing Problem master's thesis by Petr Eichler\cite{Eichler:Petr:2003}.

These instances solve the real-life vehicle routing problem 
and mainly for that reason are ideal for the testing purposes.
We slightly modified the code from the thesis for observation purposes
and added the time measuring functionality,
which tracked the time between the algorithm's iterations and the current solution value of the job in each iteration.

In overall,
we executed and measured 56 different instances of TASP.
The observations can be found in \inlinedata{jobs-data/input} folder inside the implementation project
and they are being used in the simulations,
where the simulation module randomly selects one file with runtime data for each job 
and assigns it to the job, that is being scheduled.
The data are then effectively used as input data for the load balancer.

\section{Simulations}\label{sec:simulations}

There are two main randomized simulation scenarios inside simulation module 
(module described in \ref{subsec:simulations-architecture}).
Both simulation can be started up locally using \inlinecode{SimulationExecutor} or
using \inlinecode{docker-compose} microservices runtime environment.
Input data for tests are created randomly,
based on number of scheduled jobs and given planning horizon (how much steps ahead should scheduler count with).
The number of scheduled jobs can be easily edited directly in code.
For first simulation data and input please refer to \inlinecode{OnePlanningRoundSimulation},
for second to \inlinecode{RuntimeSimulation}.

Simulation's output (produced solution) is printed to standard output.
Data displayed in listing~\ref{lst:data-example} are an output produced by simulation with 5 jobs being scheduled at once.
\medskip
\begin{samepage}
\begin{lstlisting}[caption={Simulation data output},label={lst:data-example},language=Kotlin]
                Times:||  0| 60|120|180|240|300|360|
    ----------------- || --|---|---|---|---|---|---|
        Cost: 1 + 0.02||  2|  2|  4|  3|  3|  2|  2|
        Cost: 1 + 0.02||  2|  2|  4|  3|  3|  2|  2|
        Cost: 1 + 0.02||  2|  2|  4|  3|  3|  2|  2|
        Cost: 1 + 0.02||  3|  4|  4|  3|  3|  2|  2|
        Cost: 1 + 0.02||  4|  4|  4|  3|  3|  2|  2|

      Cost: 1.5 + 0.02||  1|  1|  1|  2|   |   |   |
      Cost: 1.5 + 0.02||  1|  1|  1|  2|   |   |   |

     Cost: 10.0 + 0.05||  0|  0|   |   |   |   |   |
     Cost: 10.0 + 0.05||  0|  0|   |   |   |   |   |

    Total plan cost: CostImpl(value=321.08)$
\end{lstlisting}
\end{samepage}
\medskip \noindent
\inlinecode{Times} axis shows time units ($t$ value described in section \ref{sec:formal-definition}) in seconds.
Second, vertical axis, visualizes resources.
Name, \inlinecode{Cost: 1 + 0.02} is name of the resources provider (explained in section \ref{subsec:formalized-definition-representation}),
and each line is one usage unit - meaning that it could be one physical core or for example percentage of shared processor.
In implementation, this value is referred as CPU core,
but in fact, it is dimensionless value expressing usage of some system.

Each cell contains either number or is empty.
Number is job ID and indicates, that this resource is allocated to job with displayed ID.
This is effectively $^{r}x_{t}^{j}$ value explained in section \ref{subsec:variables-definition}.

Total plan cost is sum of all particular costs of all jobs - their resources allocations.
Therefore this is the cost of created plan.

First simulation is designed to use only first scheduling stage described in section \ref{sec:olb-algorithm}
and create static initial plan which is then not modified.
This simulation is implemented in \inlinecode{OnePlanningRoundSimulation} class.

TODO - more information about simulations