\chapter{Experiments}\label{ch:experiments}

In this chapter, 
we would like to present,
how we tested the load balancer.
The simulations were designed to reflect the real-life situations 
and to simulate the most common usage of the load balancer.


\section{Simulations implementation}\label{sec:simulations-implementations}
To test the load balancer,
there are two modules, 
that contain simulation code used for testing -
\inlinecode{simulation} and \inlinecode{remote-scheduler} module.
The simulation engine and all simulation use cases are located in the simulation module.
The second mentioned module contains server, 
which runs simulations on the remote API. 
This is especially useful when testing whole scheduling environment
and it is as closest as possible to the real-life environment,
which should be based on the microservices architecture.

Both simulation can be started up locally using \inlinecode{SimulationExecutor} or
using \inlinecode{docker-compose} inside the microservices runtime environment.
Input data for the tests are created randomly,
based on number of scheduled jobs and given planning horizon (how much steps ahead should scheduler count with).
The number of scheduled jobs can be easily edited directly in code.
For the simulation data and the input configuration please refer to \inlinecode{OnePlanningRoundSimulation} 
and \inlinecode{RuntimeSimulation} classes.

\section{Optimization algorithms data}\label{sec:optimization-algorithms-data}
For the proper testing environment,
the real runtime data of the optimization algorithm were needed.
We decided to use TASP (mentioned in subsection \ref{subsubsec:tasp}) as heuristic algorithm,
which execution could be potentially scheduled by the instance of the optimization load balancer.
We did not implement a new TASP instance,
instead,
we used examples from Stochastic Dynamic Vehicle Routing Problem master's thesis by Petr Eichler\cite{Eichler:Petr:2003}.

These instances solve the real-life vehicle routing problem 
and mainly for that reason are ideal for the testing purposes.
We slightly modified the code from the thesis for observation purposes
and added the time measuring functionality,
which tracked the time between the algorithm's iterations and the current solution value of the job in each iteration.

In overall,
we executed and measured 56 different instances of TASP.
The observations can be found in \inlinedata{jobs-data/input} folder inside the implementation project
and they are being used in the simulations,
where the simulation module randomly selects one file with runtime data for each job 
and assigns it to the job, that is being scheduled.
The data are then effectively used as input data for the load balancer.

\section{Simulations}\label{sec:simulations}

There are two main randomized simulation scenarios inside simulation module 
(module described in \ref{subsec:simulations-architecture}).
First simulation is designed to be static.
It creates only one execution plan
and then it shut downs.
The second simulation reflects the presumed production environment
and is described in the figure \ref{fig:simulation-process}.

\begin{figure}[ht] 
	\includegraphics[width=\textwidth]{i_simulation_process.pdf} 
	\centering
	\caption{Simulation process}
	\label{fig:simulation-process}
\end{figure}

\begin{itemize}
	\item \textbf{Generate jobs} - simulation randomly create new optimization jobs, 
	      which should be scheduled by the core.
	      It also generates their scheduling parameters.
	\item \textbf{Generate solution values} - simulation uses the solution values functions,
	      generated by the TASP (\ref{subsubsec:tasp}) algorithm as described in section \ref{sec:optimization-algorithms-data}.
	      The data files are randomly assigned to the particular jobs, 
	      therefore each job represents unique TASP instance
	      and has unique solution value during time.
	\item \textbf{Create execution plan} - simulation executes plan creation by calling the core API.
	      The process of plan creation is described in section \ref{sec:olb-algorithm}.
	\item \textbf{Evaluate and display plan} - simulation engine uses evaluator to check for the constraint violations
	      and then prints the result into the log.
	      It also prints the text representation of the plan into the standard output. 
	      An example of such plan in text representation can be seen in listing \ref{lst:data-example}.
\end{itemize}

\subsection{Simulation output}\label{subsec:simulation-output}

Simulation's output (produced execution plan) is printed to standard output.
Data displayed in the table~\ref{table:execution-plan} are a formalized output produced by the simulation with 10 jobs being scheduled at once.
The optimization jobs,
which were scheduled in the table~\ref{table:execution-plan},
have their parameters displayed in the table~\ref{table:jobs-parameters}. 
The parameters can be visualized by the application as well, 
please refer to the file \inlinecode{AllocationsPlanExtensions.kt}.

\begin{table}[ht]
	\centering
	\caption{Table with the input and output parameters of the jobs}
	\begin{tabular}{|c|c c c c c c c c c c|} 
		\hline
		$j$       & 0   & 1     & 2   & 3     & 4     & 5     & 6     & 7     & 8     & 9 \\
		\hline\hline
		$D^{j}$   & 638 & 650   & 373 & 371   & 519   & 624   & 621   & 407   & 322   & 725 \\
		\hline
		$P^{j}$   & 75  & 52    & 96  & 51    & 34    & 144   & 103   & 46    & 18    & 29 \\
		\hline\hline
		$T^{j}$   & 600 & 540   & 120 & 360   & 460   & 600   & 420   & 300   & 120   & 540  \\
		\hline
		$C^{j}$   & 27.4& 40.84 & 29.6& 26.54 & 19.8  & 30.54 & 15.74 & 27.78 & 14.8  & 19.3 \\
		\hline
	\end{tabular}
	\label{table:jobs-parameters}
\end{table}

The variables $D,P,T$ and $C$ are defined in section \ref{sec:formal-definition}
and represents following data related to the one job.
\begin{itemize}
	\item $D^{j}$ - maximal duration of the job execution which cannot be exceeded
	\item $P^{j}$ - maximal used resources cost per job,
	      or in other words highest possible price paid for the job execution which cannot be exceeded
	\item $T^{j}$ - time taken, duration of the actual job execution
	\item $C^{j}$ - resource costs, how much money was actually paid for the job execution
\end{itemize}

The following data output in the table \ref{table:execution-plan} is result of the first scheduling window 
(how does the load balancing algorithm work is described in section \ref{sec:olb-algorithm}).

\begin{table}[ht]
	\centering
	\caption{Simulation data output}
	\begin{tabular}{|c|c c c c c c c c c c c|} 
		\hline
		$t$:              & 0 & 1 & 2 & 3 & 4 & 5 & 6 & 7 & 8 & 9 & 10 \\
		\hline\hline
		$^{1.1}x_{t}^{j}$ & 1 & 1 & 2 & 2 & 2 & 1 & 5 & 5 & 1 & 1 & 0  \\
		$^{1.2}x_{t}^{j}$ & 9 & 4 & 6 & 2 & 3 & 3 & 7 & 5 & 8 & 9 & 8  \\
		\hline
		$^{2.1}x_{t}^{j}$ & 0 & 0 & 5 & 0 & 7 & 0 & 4 & 4 & 6 & 4 & 3  \\
		$^{2.2}x_{t}^{j}$ & 5 & 7 & 7 & 9 & 7 & 9 & 9 & 4 & 6 & 6 & 5  \\
		\hline
	\end{tabular}
	\label{table:execution-plan}
\end{table}

Time axis shows time units ($t$ value described in section \ref{sec:formal-definition}).
Second, vertical axis, visualizes resources.
$r$ value is in the format $x.y$ where the $x$ value is the resources provider (explained in section \ref{subsec:formalized-definition-representation}),
and $y$ is one usage unit - meaning that it could be one physical core or for example percentage of shared processor.
In implementation, this value is referred as CPU core,
but in fact, it is dimensionless value expressing usage of some system resources.

Each cell contains either number or is empty.
Number is job ID and indicates, that this resource is allocated to job with displayed ID.
This is effectively $^{r}x_{t}^{j}$ value explained in section \ref{subsec:variables-definition}.

Total plan cost is sum of all particular costs of all jobs - their resources allocations.
Therefore this is the cost of the created execution plan.