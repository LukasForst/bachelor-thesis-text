\chapter{Experiments}\label{ch:experiments}

To extensively test the application,
there are two modules, 
that contain simulation codes used for testing -
\inlinecode{simulation} and \inlinecode{remote-scheduler} modules.
Simulation engine and all simulation use cases are located in simulation module.
The second mentioned module contains server, 
which runs simulations on remote API. 
This is especially useful when testing whole scheduling environment for example in Docker Compose network
and it is as closest as possible to real life environment,
which should be running in micro services architecture.

\section{Optimization algorithms data}\label{sec:optimization-algorithms-data}
For the proper testing environment,
the real runtime data of the optimization algorithm were needed.
We decided to use TASP (mentioned in subsection \ref{subsubsec:tasp}) as heuristic algorithm,
which execution could be potentially scheduled by the instance of the optimization load balancer.
We did not implement a new TASP instance,
instead,
we used examples from Stochastic Dynamic Vehicle Routing Problem master's thesis by Petr Eichler\cite{Eichler:Petr:2003}.

These instances solve the real-life vehicle routing problem 
and mainly for that reason are ideal for the testing purposes.
We slightly modified the code from the thesis for observation purposes
and added the time measuring functionality,
which tracked the time between the algorithm's iterations and the current solution value of the job in each iteration.

In overall,
we executed and measured 56 different instances of TASP.
The observations can be found in \inlinedata{jobs-data/input} folder inside the implementation project
and they are being used in the simulations,
where the simulation module randomly selects one file with runtime data for each job 
and assigns it to the job, that is being scheduled.
The data are then effectively used as input data for the load balancer.

\section{Simulations}\label{sec:simulations}
Simulation module, particular simulation scenarios, scenarios execution\todo{here write how i did it}