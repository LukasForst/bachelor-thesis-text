%%
%% Author: lukas
%% 01.01.2019
%%

\chapter{Introduction}\label{ch:introduction}

\section{Problem definition}\label{sec:problem-definition}

\subsection{What I want}\label{subsec:what-i-want}

\begin{itemize}
    \item I have generic scheduling algorithm which does not need to know the domain it is operating with
    \item I have many different domains where each requires it's own visualisation application (let's call it \textit{client} application)
    \item I want to have one instance of scheduling algorithm which can perform scheduling for each client application
    \begin{itemize}
        \item This instance must be easily scalable and generic such as adding new problem domain is not \textit{"a big deal"}
    \end{itemize}
    \item I want to change as little as possible when adding a new domain (and thus new client application)
    \item I want this algorithm to run on dedicated server
\end{itemize}

\subsection{Remote scheduler application requirements}\label{subsec:remote-scheduler-application-requirements}

\begin{itemize}
    \item  Client application does not contain any scheduling algorithm
    \begin{itemize}
        \item Client contains converter (which is used to translate data from client's database model into remote scheduler data model) and \textit{http} client which
        ensures connection with remote scheduling server
    \end{itemize}
    \item When client wants to create new plan, it just transforms database data into scheduler data model and sends them to
    \textbf{RMS}\footnote{\textbf{R}emote\textbf{S}cheduler \textbf{S}erver}
    \item RMS is modified as little as possible while adding new client application (and thus new problem domains)
    \begin{itemize}
        \item Core scheduling algorithm \textbf{must not} be changed
        \item No code can be changed when adding new domain
        \begin{itemize}
            \item Only new domain-specific behaviour can be added such as heuristics, plan evaluation etc.
            \item It must be ensured that these domain specific parts of RMS will be used only while scheduling correct domain
        \end{itemize}
    \end{itemize}
    \item RMS ensures that clients scheduling data/algorithms can't interact with each other
    \begin{itemize}
        \item Each scheduling must be enclosed and exclusively accessible only for client that started it
        \item Domain specific code can't be used in different domain
    \end{itemize}
    \item RMS is able to balance it's load
    \begin{itemize}
        \item It is possible to outsource SJ\footnote{SJ - Scheduling Job} to another instance of RMS
        \item SJ have priority and assigned resources, when priority is set to low, resources can be reassigned to SJ with higher priority
    \end{itemize}
    \item RMS supports different scheduling configuration
    \begin{itemize}
        \item SJ can have different requirements on RAM/CPU/IO
        \item SJ can have different complexity and thus different scheduling phases have different performance requirements
        \begin{itemize}
            \item ie.
            inserting assignments to final plan, removing assignments from plan, plan evaluation etc.
        \end{itemize}
    \end{itemize}
\end{itemize}

\section{Motivation to solve it}\label{sec:motivation-to-solve-it}

