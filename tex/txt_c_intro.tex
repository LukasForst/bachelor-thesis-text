\chapter{Introduction}\label{ch:introduction}
The globalization of the world’s economies is a major challenge to local industry 
and it is pushing the manufacturing sector to the transformation called \textit{Industry 4.0} \cite{industry40}.
In order to become more competitive, 
manufacturers need to embrace emerging technologies, 
such as advanced analytics, artificial intelligence 
and mathematical optimization to improve their efficiency and productivity.

Specifically manufacturing industry sector,
which have high production costs,
faces multiple problems,
where employing mathematical optimization can reduce cost or improve efficiency of the process.
Taking as en example the car manufacturers,
there are many processes,
that can be optimized to reduce their cost or to use needed resources more efficiently,
such as an internal logistics, car parts transportation, parts stocking, cars manufacturing 
and the allocation of the various types of resources.
These optimization challenges are often solved by the proprietary software systems with included optimization engine,
where the one problem domain is usually handled by the single program operating specifically with such domain.

These applications typically have a user interface for the data visualization
and an engine running an optimization algorithm.
Although the data visualization part of the application does not require powerful hardware,
the immense complexity of the mathematical optimization problems 
and thus the performance requirements for such optimization engine solving them, are not always satisfiable.
Moreover,
the computer performance is finite resource
and it cost money paid for the computer components or for the electricity used.
For that reasons,
the optimization software systems have often limited access to the computer resources.
In addition,
this computer performance is not used all the time,
since the typical usage of such application lays mainly in data visualization,
hence the performance, required only when the optimization engine is running, is unused.

Using such software architecture seems to be highly inefficient,
since the instances, that are in time $t$ running the optimization algorithm, are overwhelmed 
and in the same time $t$,
the applications, that are not running the optimization tasks,
do not use their powerful computer resources at all.

The potential solution for this problems lays in microservices architecture,
where the parts of the applications are independent and able to run separately.
Using this approach enables distributed computing
and therefore outsourcing the demanding optimization engine to more powerful servers.
This approach solves the lack of resources for the optimization engine on the less powerful servers,
but it also introduce new challenge in the load balancing of the servers,
where the optimization engine is being executed.

In this thesis we would like to present the load balancer specifically developed for the optimization algorithms,
which should minimize resources wasting 
and increase the performance using correct utilization distribution across the multiple instances of such algorithms.

\section{Thesis goals}\label{sec:thesis-goals}

This thesis extensively studies and describes state of the art algorithms used for computational task scheduling
and load balancing in section \ref{sec:load-balancing}. 
It also brings overview of various optimization problems, techniques and algorithms 
that are being used in real-life situations to solve diverse industries problems 
in section \ref{sec:optimization-algorithms}.
The complex problem of computational task scheduling is formalized in chapter \ref{ch:problem-formalization}.

Although the main goal is to develop fully functional scheduling system,
mainly because of estimated complexity,
this thesis focuses only on system's core - 
scheduling and load balancing for optimization algorithms in heterogenous distributed environment.

Solution design including final algorithm is proposed in chapter \ref{ch:solution-design}.
In the following chapter (\ref{ch:implementation}),
thesis describes resulting implementation of the load balancing system
and also gives overview of its architecture in the section \ref{sec:architecture}.
Scheduler is then evaluated using multiple simulations in section \ref{sec:simulations}.

To achieve ultimate goal, having fully functional scheduling and load balancing system,
future necessary steps are outlined in section \ref{sec:future-work}.
