%%
%% Author: lukas
%% 01.01.2019
%%

\chapter{Introduction}\label{ch:introduction}
\todo{Whole text does not seems to be right, maybe I will need to rewrite it.}
Optimization algorithms and solutions build on them are widely used in current manufacturing industry to reduce production costs.
With more and more production automatization, optimization algorithms can manage and schedule whole factories with maximum available efficiency.

Complexity of optimization problems could be huge and therefore performance requirements are sometimes not easily satisfiable.
Using one powerful instance of optimization algorithm in cloud seems like a solution for problems with smaller complexity,
but what if we have multiple huge problems where each is performance demanding?
Of course, we can create multiple instances, but that would be expensive and not well manageable and scalable
since adding another instances manually requires some time and it is not much flexible.
Another disadvantage of this approach is the fact, that optimization algorithm is not running $100\%$ of time
and thus resources allocated by this algorithm are unused while other algorithm instances could be potentially overwhelmed.
Also paying for unused hardware is wasting money and optimization algorithms are supposed to save money.

Now imagine having two completely different problems that each requires its own application which visualises data
and optimization algorithm to compute some kind of plan,
this algorithm can be generic enough to operate on both domains with same code base, but it requires a lot of performance resources.
If we use monolithic architecture of both applications,
we would have same code in two applications,
but what is even worse, we would need two powerful machines to run our applications.
As previously mentioned, these two machines would not be using their power whole time and would be mainly idle.

What if one application runs only few minutes a day, but needs that power to complete tasks in time?
A lot of resources would be wasted if it has its own server,
but using not powerful server would lead to increasing duration of ongoing tasks which is something we do not want.

In this paper I would like to introduce \textbf{load balancer} specifically developed for optimization algorithms
which could potentially minimize resources wasting and increase performance using correct utilization distribution across
multiple instances of optimization algorithms.


%%
%% Author: lukas
%% 03.01.2019
%%

\chapter{Problem definition}\label{ch:problem-definition}

The problem with implementation of optimization algorithms in applications is that their
performance requirements are quite high and are fully utilized only while working.
Optimization algorithm is not running all the time and for that reason hardware resources are mainly unused.
These unused resources could be potentially used by another instance of algorithm
or can be shutdown completely to reduce hosting costs.

Also adding more time to job execution does not always bring better solution
but it certainly costs more.
Therefore proposed load balancer must be able to stop execution when solution value
is not getting better compared with scheduling costs.

%%
%% Author: lukas
%% 12.02.2019
%%

\section{Formal definition}\label{sec:formal-definition}
Whole definition implies that it can be represented as Integer Linear Programming (ILP).
However, it is not entirely formalized, 
it is only mechanical work to transcript all implications into equations.
For that reason, and better readability, I left the implications.

\subsection{Variables Definition}\label{subsec:variables-definition}
Following indexes, inputs and variables are used in the optimization criteria.

\subsubsection{Indexes}
\begin{itemize}
	\item $j$ - index used to identify something related to the execution job,
	      in real world this is most likely job id, located in the right upper corner - $x^{j}$, 
	      set of all jobs in the system is represented by $J$
	\item $r$ - index used to identify resources, written in the left upper corner - ${}^{r}x$, set of resources is represented by $R$
	\item $t$ - right bottom index represents time - $x_t$
\end{itemize}

\subsubsection{Input}\label{subsubsec:formal-input}
Input which is specified before executing optimization job by the user outside of the system.
When new execution job is requested - this is done by the user, or client application,
following data should be provided.
\begin{itemize}
	\item $D^{j}$ - maximal duration of the job execution which cannot be exceeded
	\item $P^{j}$ - maximal used resources cost per job, or in other words highest possible price paid for the job execution which cannot be exceeded
\end{itemize} 
There are also constants defined before load balancer start up.
\begin{itemize}
	\item ${}^{r}c$ - cost of the using particular resources per one time unit
\end{itemize}
Each of the previously mentioned variable must be non-negative.

\subsubsection{Program Output}
Apart from result of the underlying optimization algorithm,
following data are returned to user after successful job execution.

\begin{itemize}
	\item $T^{j}$ - time taken, duration of the actual job execution
	\item $C^{j}$ - resource costs, how much money was actually paid for the job execution
\end{itemize}

\subsubsection{Variables}

\begin{itemize}
	\item $v_{t}^{j}$ - value\todo{better example would be fine} (i.e.\ cost of the scheduled plan) of the job $j$ at the time $t$
	      Value is greater than zero and it is non-increasing during the time. 
	      \begin{equation}
	      	\forall t, j: v_{t}^{j} \geq v_{t+1}^{j} > 0 
	      \end{equation}
	      It is non-increasing because optimization algorithms return always best found solution, 
	      so when worse solution, than currently best one, is found,
	      returned is still the best solution found.
	\item $^{r}x_{t}^{j}$ - represents assignment of the resources $r$ at time $t$ to job $j$
	      \begin{equation}
	      	^{r}x_{t}^{j} = \{0, 1\} 
	      \end{equation}
	      Each $x$ is either $1$ = indexed resources are assigned to the job at given time or $0$ = given combination does not have assignment.
	      We assume that each job has only one such assignment at one time,
	      which effectively means that this job is executed on the single computation node.
	      This is defined by following constraint:
	      \begin{equation}
	      	\forall j, t:\; \sum_{r \in R} {}^{r}x_{t}^{j} \leq 1 
	      \end{equation}
	\item $^{r}\Delta_{t}^{j}$ - enhancement of the value $v$ with resources $r$ on the job $j$ per time $t$.
	      \begin{equation}
	      	^{r}\Delta_{t}^{j} = {}^{r}| v_{t}^{j} - v_{t-1}^{j}| \cdot {}^{r}x_{t}^{j}
	      \end{equation}
	      It is improvement of the solution value $v$ which can be achieved by using resources $r$ at time $t$.
	      This value is always non-negative since optimization algorithm always stores best found solution,
		  and therefore $\forall j, t, r:\; {}^{r}\Delta_{t}^{j} \geq 0$. 
		  ${}^{r}x_{t}^{j}$ ensures that only resources, 
		  that are actually used, are taken in account.
	\item $S_{t}^{j}$ - reward for improving solution value until time $t$ per job $j$.
	      Accumulation of enhancements $^{r}\Delta_{t}^{j}$ through all resources $r$ and time units $t$.
	      \begin{equation}
	      	S_{t}^{j} = \sum_{t}\:\sum_{r \in R}\: {}^{r}\Delta_{t}^{j} 
	      \end{equation}
	\item $C_{t}^{j}$ - defines how much execution of job $j$ cost from the beginning of the execution until time $t$.
	      Sum of all allocated resources for their time for the particular job.
	      \begin{equation}
	      	C_{t}^{j} = \sum_{t}\:\sum_{r \in R}\: {}^{r}c \cdot {}^{r}x_{t}^{j} 
	      \end{equation}
	      Because $C_{t}^{j}$ is defined as sum and ${}^{r}c$ is non-negative,
	      it is true that $\forall j, t:\; C_{t+1}^{j} \geq C_{t}^{j}$.
	      Input of the program specifies maximal cost paid for job execution as a $P^j$, 
	      therefore it must be enforced by the system that this cost will not be exceeded.
	      This constraint can be defined as follows.
	      \begin{equation}
	      	\forall t, j:\; P^{j} \leq C_{t}^{j} \implies \sum_{t+1}^{\infty} \, \sum_{r \in R} {}^{r}x_{t}^{j} = 0 
	      \end{equation}
	      Which effectively means that when cost of job execution $C_{t}^{j}$ has reached maximal defined cost $P^{j}$,
	      no resources can be assigned to this job.
	\item $t$ - time, it is not only index but also variable, there are also constraints regarding time -
	      since client application can specify deadline to job $D^{j}$,
	      there must be additional constraint for job execution in a matter of resources assignment.
	      \begin{equation}
	      	\forall t, j:\; D^{j} \leq t \implies \sum_{t+1}^{\infty} \, \sum_{r \in R} {}^{r}x_{t}^{j} = 0 
	      \end{equation}
	      When maximal time is exceeded, no additional resources can be assigned to the job execution, 
	      which could be defined by following constraint.
\end{itemize}

\subsection{Resources reconfiguration}\label{subsec:resource-config}
System should be capable of changing resources assignment per job in the runtime.
This will help to distribute performance according to the current nodes load across whole network.
Unfortunately, it is not always possible to reconfigure resources assignment while scheduling is being performed.
Therefore there must be at least one time unit, between different resources assignment,
where no resources are assigned to the job.
In other words, if resources reconfiguration is triggered in time $t$ then $\sum_{r \in R} {}^{r}x_{t}^{j} = 0$.
This can be formalized as:
\begin{equation}
	\sum_{r \in R} {}^{r}x_{t}^{j} = 1 \implies \forall r \in R:\; {}^{r}x_{t}^{j} - {}^{r}x_{t + 1}^{j} \geq 0
\end{equation}

\subsection{Optimization criteria}\label{subsec:optimization-criteria}
The main goal of the system is to minimize outcome value of the underlining optimization algorithm
and at the same time to minimize cost of used resources.
We can optimize single job or sum of outcomes from all jobs in the system at once.
First approach provides possibility to control and optimize outcome of a particular job,
which is an advantage for single client (job owner),
but it does not necessarily means that it is optimal for the whole system and vice versa.
Optimization criteria for the single job at particular time $t$ is then
maximization of weighted difference between the value enhancement reward 
and cost paid for the enhancement,
which can be described by following equation.
\begin{equation}
	\max crit_{t}^{j} = \alpha S_{t}^{j} - (1 - \alpha) C_{t}^{j} \qquad 0 \leq \alpha \leq 1 
\end{equation}
For optimization of system-wide resources and costs,
all jobs execution optimization is then defined like a
weighted sum of all rewards per jobs lowered by sum of all resources costs across the set of all jobs.
\begin{equation}
	\max crit_{t} = \alpha \sum_{j \in J} S_{t}^{j} - (1 - \alpha) \sum_{j \in J} C_{t}^{j} \qquad 0 \leq \alpha \leq 1 
\end{equation}
Based on the previous equation, 
it is possible to define time independent optimization criterion.
\begin{equation}
	\max crit = \sum_{t}^{\infty} crit_{t}
\end{equation}

%%
%% Author: lukas
%% 03.01.2019
%%

\section{Motivation to solve it}\label{sec:motivation-to-solve-it}
Here comes some motivation to solve it
\todo{Maybe bachelor degree?
Or at least unassisted project would be fine\ldots}
\todo{Lost in the context\ldots}

%TODO how we can save costs

%%
%% Author: lukas
%% 03.01.2019
%%

\section{Motivation to solve it}\label{sec:motivation-to-solve-it}
Here comes some motivation to solve it
\todo{Maybe bachelor degree?
Or at least unassisted project would be fine\ldots}
\todo{Lost in the context\ldots}

%TODO how we can save costs

