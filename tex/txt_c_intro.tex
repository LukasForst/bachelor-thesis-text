%%
%% Author: lukas
%% 01.01.2019
%%

\chapter{Introduction}\label{ch:introduction}
\todo{Whole text does not seems to be right, maybe I will need to rewrite it.}
Optimization algorithms and solutions build on them are widely used in current manufacturing industry to reduce production costs.
With more and more production automatization, optimization algorithms can manage and schedule whole factories with maximum available efficiency.

Complexity of optimization problems could be huge and therefore performance requirements are sometimes not easily satisfiable.
Using one powerful instance of optimization algorithm in cloud seems like a solution for problems with smaller complexity,
but what if we have multiple huge problems where each is performance demanding?
Of course, we can create multiple instances, but that would be expensive and not well manageable and scalable
since adding another instances manually requires some time and it is not much flexible.
Another disadvantage of this approach is the fact, that optimization algorithm is not running $100\%$ of time
and thus resources allocated by this algorithm are unused while other algorithm instances could be potentially overwhelmed.
Also paying for unused hardware is wasting money and optimization algorithms are supposed to save money.

Now imagine having two completely different problems that each requires its own application which visualises data
and optimization algorithm to compute some kind of plan,
this algorithm can be generic enough to operate on both domains with same code base, but it requires a lot of performance resources.
If we use monolithic architecture of both applications,
we would have same code in two applications,
but what is even worse, we would need two powerful machines to run our applications.
As previously mentioned, these two machines would not be using their power whole time and would be mainly idle.

What if one application runs only few minutes a day, but needs that power to complete tasks in time?
A lot of resources would be wasted if it has its own server,
but using not powerful server would lead to increasing duration of ongoing tasks which is something we do not want.

In this paper I would like to introduce \textbf{load balancer} specifically developed for optimization algorithms
which could potentially minimize resources wasting and increase performance using correct utilization distribution across
multiple instances of optimization algorithms.


%%
%% Author: lukas
%% 03.01.2019
%%

\chapter{Problem definition}\label{ch:problem-definition}

The problem with implementation of optimization algorithms in applications is that their
performance requirements are quite high and are fully utilized only while working.
Optimization algorithm is not running all the time and for that reason hardware resources are mainly unused.
These unused resources could be potentially used by another instance of algorithm
or can be shutdown completely to reduce hosting costs.\\
Also adding more time to job execution does not always bring better solution
but it certainly costs more.
Therefore proposed load balancer must be able to stop execution when solution value
is not getting better compared with scheduling costs.

\section{Formal definition}\label{sec:formal-definition}
\begin{itemize}
    \item $T_{\max}$ - maximal optimization job execution time provided by user and specified before execution started

    \item $T$ - actual optimization job execution time, when no execution time optimization is being used $T = T_{\max}$

    \item $RC$ - \textit{Resource Costs} - all hardware costs used for executing optimization job by some algorithm
    \todo{don't know how to say that - costs that you actually pay for hardware}
    \begin{equation}
        RC = \sum_{i=0}^T RC_i
    \end{equation}

    \item $RC_t$ - \textit{Resource Costs} in time $t$ - accumulated costs from beginning of execution to time $t$
    \begin{equation}
        RC_{t} = \sum_{i=0}^{t-1} RC_i
    \end{equation}

    \item $RC_{\max}$ - maximal resource costs specified by user in advance
    \begin{equation}
        RC_{\max} \geq RC
    \end{equation}

    \item $V$ - \textit{Solution Value} - value of the found solution, since this paper focus on cost optimization,
    \textit{Solution Value} is cost of found solution
    \begin{equation}
        V = min \{ V_t \}, \quad t = 0 \dots T
    \end{equation}

    \item $V_t$ - \textit{Solution Value} in time $t$ - best solution provided by algorithm since the beginning of the job execution
    until time $t$

\end{itemize}

Then load balancer optimizes following function

\begin{equation}
    min \{ \alpha RC + \beta V \,|\, \alpha, \beta \in \mathbb{R} \}
\end{equation}

Where $\alpha$ and $\beta$ are coefficients that are balancing $RC$ and $V$.



\subsection{Load Balancer Requirements}\label{subsec:load-balancer-requirements}
%TODO
% For successful server implementation we must enforce following things
% \begin{itemize}
%    \item Running optimization algorithms can't interfere with each other
%    \item RMS guarantee that scheduled job will be always executed
%    \item
% \end{itemize}


%%
%% Author: lukas
%% 03.01.2019
%%

\section{Motivation to solve it}\label{sec:motivation-to-solve-it}
%TODO how we can save costs

