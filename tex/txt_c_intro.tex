\chapter{Introduction}\label{ch:introduction}
The globalization of the world’s economies is a major challenge to local industry 
and it is pushing the manufacturing sector to the transformation called \textit{Industry 4.0} \cite{industry40}.
In order to become more competitive, 
manufacturers need to embrace emerging technologies, 
such as advanced analytics, artificial intelligence 
and mathematical optimization to improve their efficiency and productivity.

Specifically the manufacturing industry sector,
which have high production costs,
faces multiple problems,
where employing mathematical optimization can reduce cost or improve the efficiency of the process.
Taking as an example the car manufacturers,
there are many processes,
that can be optimized to reduce their cost or to use needed resources more efficiently,
such as internal logistics, car parts transportation, parts stocking, cars manufacturing 
and the allocation of the various types of resources.
These optimization challenges are often solved by the proprietary software systems with included optimization engine,
where the one problem domain is usually handled by the single program operating specifically with such domain.

These applications typically have a user interface for the data visualization
and an engine running an optimization algorithm.
Although the data visualization part of the application does not require powerful hardware,
the immense complexity of the mathematical optimization problems 
and thus the performance requirements for such optimization engine solving them, are not always satisfiable.
Moreover,
the computer performance is a finite resource
and it costs money paid for the computer components or for the electricity used.
For those reasons,
the optimization software systems have often limited access to computer resources.
In addition,
this computer performance is not used all the time,
since the typical usage of such application lays mainly in data visualization,
hence the performance, required only when the optimization engine is running, is unused.

Using such software architecture seems to be highly inefficient,
since the instances, that are in time $t$ running the optimization algorithm, are overwhelmed 
and at the same time $t$,
the applications, that are not running the optimization tasks,
do not use their powerful computer resources at all.

The potential solution for these problems lays in microservices architecture,
where the parts of the applications are independent and able to run separately.
Using this approach enables distributed computing
and therefore outsourcing the demanding optimization engine to more powerful servers.
This approach solves the lack of resources for the optimization engine on the less powerful servers,
but it also introduces a new challenge in the load balancing of the servers,
where the optimization engine is being executed.

In this thesis, we would like to present the load balancer specifically developed for the optimization algorithms,
which should minimize resources wasting 
and increase the performance using correct utilization distribution across the multiple instances of such algorithms.
In the first chapters,
the thesis analyzes the state-of-the-art solutions and technologies,
that are being used to solve load balancing problems in the large infrastructures.
These technologies are evaluated 
and the new domain-specific load balancing solution is proposed.
The thesis then transforms and formalizes the load balancing optimization problem into an integer linear programming problem.
The related challenges arising from the problem formalization, such as hyperbola time series prediction, are defined 
and the solutions for such problems are proposed.
Subsequently, the thesis designs and evaluates the microservices architecture for the load balancer 
and also delivers the complete implementation of the load balancer.
This thesis also proposed and implemented the simulations and experiments, 
which evaluates the implemented load balancer.

In the last chapter,
the future work is addressed 
and the steps, for moving the load balancer into the production environment, are proposed.
Apart from the planned steps for the load balancer system,
the last chapter outlines the future of the developed tools and libraries.

\section{Thesis goals}\label{sec:thesis-goals}
The main goals of the thesis were set as follows.
Apart from the original goals of the thesis,
which were set in the assignment,
many out of scope and future goals arose during the formalization, research and implementation.
These goals and an overviews how to achieve them are outlined in the section \ref{sec:future-work}.

\subsubsection{Study the state-of-the-art approach to computational tasks scheduling}
This thesis extensively studies and describes state of the art algorithms used for computational task scheduling
and load balancing in section \ref{sec:load-balancing}. 
It also presents technologies used, to solve such problems.
The difference between the state-of-the-art load balancing strategies
and why the thesis proposes and implements the new technology instead of using the existing one
is presented in section \ref{subsec:load-balancing-for-optimization-algorithms}.

\subsubsection{Study various types of optimization problems and approaches and understand their computational needs}
The thesis brings an overview of and describes various optimization problems, techniques and algorithms 
that are being used in the real-life situations to solve diverse industries problems 
in section \ref{sec:optimization-algorithms}.

\subsubsection{Study the distributed scalable architecture and approaches to schedule tasks on such architecture}

The distributed architecture study is presented in section \ref{sec:architecture}.
This section also introduces the overall distributed architecture of the load balancer
including the simulation module.

\subsubsection{Design a scheduling and load-balancing module able to ingest various optimization tasks and schedule them with respect to several criteria}
The complex problem of the load balancing of the optimization tasks is formalized in chapter \ref{ch:problem-formalization}.
Based on the problem formalization,
solution design including the final load balancing algorithm is introduced in chapter \ref{ch:solution-design}.

\subsubsection{Implement the scheduler}
The load balancing module design from chapter \ref{ch:solution-design} is used for 
load balancing module implementation in chapter \ref{ch:implementation}.
The very same chapter contains an overview of the module architecture in the section \ref{sec:architecture}.
The architecture was designed to keep future infrastructure development in mind 
and thus few out of scope future steps were outlined in section \ref{sec:future-work}.

\subsubsection{Evaluate the scheduler on a number of scenarios}
The chapter \ref{ch:experiments} describes the way,
how the scheduler and the load balancing system was tested.
It also describes in section \ref{sec:optimization-algorithms-data} how the runtime data of optimization algorithms were collected.
The next section \ref{sec:simulations} presents the scheduler evaluation.
