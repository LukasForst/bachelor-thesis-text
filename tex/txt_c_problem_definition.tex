%%
%% Author: lukas
%% 03.01.2019
%%

\chapter{Problem definition}\label{ch:problem-definition}

The problem with implementation of optimization algorithms in applications is that their
performance requirements are quite high and are fully utilized only while working.
Optimization algorithm is not running all the time and for that reason hardware resources are mainly unused.
These unused resources could be potentially used by another instance of algorithm
or can be shutdown completely to reduce hosting costs.

Also adding more time to job execution does not always bring better solution
but it certainly costs more.
Therefore proposed load balancer must be able to stop execution when solution value
is not getting better in comparison with scheduling costs.



%%
%% Author: lukas
%% 12.02.2019
%%

\section{Detailed formal definition}\label{sec:formal-definition-detailed}
Following detailed formal definition aims to record and model all related actions and variables in the whole optimization load balancing system.

\subsection{Variables definition}\label{subsec:detailed-variables-definition}

\subsubsection{Indexes}
\begin{itemize}
	\item $j$ - index used to identify something related to the execution job, in real world this is most likely job id
	\item $p$ - index used to identify particular resources provider (for example single computation node in local network
	      or \textit{AWS}\footnote{\textit{Amazon Web Services} is a subsidiary of Amazon that provides on-demand cloud computing platforms} instance)
	\item $a$ - index for identification of particular algorithm (i.e.\ \textit{GLPK}\footnote{\textit{GNU Linear Programming Kit} is a software package intended for solving large-scale linear programming (LP),
	      	or \textit{TASP}\footnote{\textit{Task and Asset Scheduling Platform} - proprietary optimization software developed by Blindspot Solutions, described in\ref{subsubsec:tasp}}
	      	mixed integer programming (MIP), and other related problems, described in\ref{subsubsec:glpk}})
\end{itemize}

\subsubsection{Input}
Input which is specified before executing optimization job by user outside of the system.

\begin{itemize}
	\item $T_{\max}^{j}$ - maximal duration of the job execution which cannot be exceeded
	\item $C_{\max}^{j}$ - maximal used resources cost per job, or in other words highest possible price paid for the job execution which cannot be exceeded
	\item $a$ - algorithm which should be used to run optimization
	\item $d^{j}$ - input data for the algorithm
\end{itemize}

\subsubsection{Program output}
Following data are returned back to user after successful job execution.
 
\begin{itemize}
	\item $S^{j}$ - problem solution provided by algorithm $a$, i.e.\ planned data
	\item $V^{j}$ - solution value provided by algorithm $a$
	\item $T^{j}$ - time taken, duration of the actual job execution
	\item $C^{j}$ - resource costs, how much job execution cost
\end{itemize}

\subsubsection{Time}
In this detailed problem definition, 
time is represented as series of $moments$.\todo{Maybe definition should be slightly different}
Each moment represents time period from the time $t_i$ to time $t_{i+1}$,
$moment$ is then written as $m_i$.
\begin{align*}
	|m_{i}| = t_{i+1} - t{i} 
\end{align*}
Also, each $moment$ is defined for the one job and it is index by $j$.
$m_{i}^{j}$ is an example of one moment $i$ which occurred during the executing job $j$.\\
$M^{j}$ is the count of all moments, that occurred during the job $j$ execution.
It is a fact that:
\begin{align*}
	T^{j} = \sum_{i = 0}^{M^{j}} | m_{i}^{j} | 
\end{align*}
Or in other words, total execution time of the job is sum of lengths of moments, that were part of the job execution.

New moment must be created when resources assigned to the job are changed.
But it is possible to create new moment without the resource change.

\subsection{Functions}\label{subsec:detailed-functions}

There are three main functions which are used in mathematical description of the system.

\subsubsection{Solution cost}
This function defines how much cost (in money) resource allocation for particular job
and it is effectively used to express cost dependence on resources and time in any particular moment of the job execution.
\begin{align*}
	c_{m_{i}^{j}}^{j} = g_{p} (\, |m_{i}^{j}|,\, R_{m_{i}^{j}}^{j} \,) 
\end{align*}
Where function $g_{p}$ defines how much cost resources $R_{m_{i}^{j}}^{j}$ allocation for time $|m_{i}^{j}|$ using resources provider $p$.
It is defined for moment $m_{i}^{j}$ and job $j$.\\
Therefore it is now possible to express final resource cost per job $C^{j}$.
\begin{align*}
	C^{j} = \sum_{i = 0}^{M} g_{p} (\, |m_{i}^{j}|, R_{m_{i}^{j}} \,) 
\end{align*}

\subsubsection{Solution value}
In order to compute solution value we need two functions.
One for value computation itself and one which will define,
how we get data to compute such solution value.

Let's define new variable $s$ which represents partial solution of the optimization problem.
This partial solution depends on time - with increasing time, solution is being changed, more optimized -
and it is dependent on the job $j$ - each job has its own solution.
For that reason definition of solution is $s_{m_{i}}^{j}$.

Partial solution is computed by the algorithm,
its value depends on the duration of the execution,
on provided data and on used computation resources.
Generic solution therefore looks like this:
\begin{align*}
	s & = f_a(t, R, d) 
\end{align*}
The function $f_{a}$ is defined as \textit{the ability of algorithm $a$ to improve solution $d$ with used resources $R$ and time $t$ to new solution $s$}.

The solution $s$ consists of two parts, data used for computation and found solution - $s = [\text{partial solution}, \text{data}]$.
Since $d$ and $s$ have same type, we can write it indexed - because $s$ is based on iteration made over $d$.
Also the function arguments are time dependent - moment index is needed.
The final function $f_{a}$ is defined as:\todo{Pay attention to indexing -> maybe I will need to change it.}
\begin{align*}
	s_{m_{i+1}}^{j} & = f_{a}(\, |m_{i}^{j}|, R_{m_{i}^{j}}^{j}, s_{m_{i}^{j}}^{j}\,) \\
\end{align*}
And for the first algorithm iteration:
\begin{align*}
	s_{m_{0}}^{j} & = f_{a}(\, |m_{0}^{j}|, R_{m_{0}^{j}}^{j}, d^{j}\,) 
\end{align*}
Where $d$ are first data provided by user as an input of the program.
The function $f_{a}$ only provides a way,
how solution is being produced but it does not define how the solution should be evaluated.
For that reason another evaluation function is needed.

\bigskip

\noindent Function $g_{a}$ defines actual value of provided solution $s^{j}$.
\begin{align*}
	v_{m_{i}^{j}}^{j} & = h_{a}(\, s_{m_{i}^{j}}^{j} \,) 
\end{align*}
Where variable $v_{m_{i}^{j}}^{j}$ represents solution value $s$ of the job $j$ in the moment $m_{i}$.
We assume, that after each iteration of algorithm better or at least same solution value is returned
and function $h_{a}$ is for the job $j$ non-ascending over moments $m_{i}^{j}$.
\begin{align*}
	h_{a}(\, s_{m_{i+1}^{j}}^{j} \,) \leq h_{a}(\, s_{m_{i}^{j}}^{j} \,) 
\end{align*}
This assumption can be made simply because when multiple feasible solutions of optimization problem are found,
algorithm always returns the cheapest one.\todo{Does this apply always?}

Because function $h_{a}$ is non-ascending,
its optimal value is located in the last moment $M^j$ of the time series $m_{0}^{j} \dots m_{M^{j}}^{j}$.
\begin{align*}
	V^{j} = h_{a}(\, s_{m_{M^{j}}^{j}}^{j} \,) 
\end{align*}

\subsection{Optimization criteria}\label{subsec:detailed-optimization-criteria}
This leads to two optimization criteria, 
where the system would be looking for the "ideal" end $moment$ $M$ 
and series of configurations $R$ such as final value $V$ and $C$ is minimal.

\begin{align*}
	\min V^{j} & = h_{a}(\, s_{m_{M^{j}}^{j}}^{j} \,) = h_{a}(\, f_{a}(\, |m_{M^{j}}^{j}|, R_{m_{M^{j}}^{j}}^{j}, s_{m_{M^{j}}^{j}}^{j}\,) \,) \\
	\min C^{j} & = \sum_{i = 0}^{M} g_{p} (\, |m_{i}^{j}|, R_{m_{i}^{j}} \,)                                                                   
\end{align*}

\bigskip

\noindent Unfortunately, \todo{add more reasons why not to use this particular definition}
this definition seems to be a bit confusing
and therefore I present simplified definition, or another approach to the problem.

\bigskip

\noindent Unfortunately, \todo{add more reasons why not to use this particular definition}
this definition seems to be a bit confusing and it later in the paper it led to problems while defining system's behavior.

And thus I present simplified definition - another approach to the problem.

%%
%% Author: lukas
%% 12.02.2019
%%

\section{Simplified formal definition}\label{sec:formal-definition-simplified}
This more general and simplified formal definition covers main parts of the optimization problem 
and does not use excessive definition of all possible variables.

It treats resources as homogenous set of all possible resource combination,
therefore resources set contains all possible combination of CPU/RAM configuration that is available.
Thanks to this simplification, resources assignment is can be binary,
where one represents assignment and zero represents no assignment.

It also omits data variable as well as underlying algorithm and resources provider index, 
in favor of better readability.


\subsection{Variables Definition}\label{subsec:variables-definition}
Following indexes, inputs and variables are used in the optimization criteria.

\subsubsection{Indexes}
\begin{itemize}
	\item $j$ - index used to identify something related to the execution job,
	      in real world this is most likely job id, located in the right upper corner - $x^{j}$, 
	      set of all jobs in the system is represented by $J$
	\item $r$ - index used to identify resources, written in the left upper corner - ${}^{r}x$, set of resources is represented by $R$
	\item $t$ - right bottom index represents time - $x_t$
\end{itemize}

\subsubsection{Input}
Input which is specified before executing optimization job by the user outside of the system.
When new execution job is requested - this is done by the user, or client application,
following data should be provided.
\begin{itemize}
	\item $D^{j}$ - maximal duration of the job execution which cannot be exceeded
	\item $P^{j}$ - maximal used resources cost per job, or in other words highest possible price paid for the job execution which cannot be exceeded
\end{itemize} 
There are also constants defined before load balancer start up.
\begin{itemize}
	\item ${}^{r}c$ - cost of the using particular resources per one time unit
\end{itemize}
Each of the previously mentioned variable must be non-negative.

\subsubsection{Program Output}
Apart from result of the underlying optimization algorithm,
following data are returned to user after successful job execution.

\begin{itemize}
	\item $T^{j}$ - time taken, duration of the actual job execution
	\item $C^{j}$ - resource costs, how much job execution cost
\end{itemize}

\subsubsection{Variables}

\begin{itemize}
	\item $v_{t}^{j}$ - value\todo{better example would be fine} (i.e.\ cost of the scheduled plan) of the job $j$ at the time $t$
	      Value is greater than zero and it is non-increasing during the time. 
	      \begin{align*}
	      	\forall t, j: v_{t}^{j} \geq v_{t+1}^{j} > 0 
	      \end{align*}
	      It is non-increasing because optimization algorithms return always best found solution, 
	      so when worse solution, than currently best one, is found,
	      returned is still the best solution found.
	\item $^{r}x_{t}^{j}$ - represents assignment of the resources $r$ at time $t$ to job $j$
	      \begin{align*}
	      	^{r}x_{t}^{j} = \{0, 1\} 
	      \end{align*}
	      Each $x$ is either $1$ = indexed resources are assigned to the job at given time or $0$ = given combination does not have assignment.
	      We assume that each job has only one such assignment at one time,
	      which effectively means that this job is executed on the single computation node.
	      This is defined by following constraint:
	      \begin{align*}
	      	\forall j, t:\; \sum_{r \in R} {}^{r}x_{t}^{j} = \{0, 1\} 
	      \end{align*}
	\item $^{r}\Delta_{t}^{j}$ - enhancement of the value $v$ with resources $r$ on the job $j$ per time $t$.
	      \begin{align*}
	      	^{r}\Delta_{t}^{j} = {}^{r}| v_{t}^{j} - v_{t-1}^{j}| 
	      \end{align*}
	      It is improvement of the solution value $v$ which can be achieved by using resources $r$ at time $t$.
	      This value is always non-negative since optimization algorithm always stores best found solution,
	      and therefore $\forall j, t, r:\; {}^{r}\Delta_{t}^{j} \geq 0$. 
	\item $S_{t}^{j}$ - reward for improving solution value in time $t$ per job $j$.
	      Accumulation of enhancements $^{r}\Delta_{t}^{j}$ through all resources $r$ and time units $t$.
	      \begin{align*}
	      	S_{t}^{j} = \sum_{i = 0}^{t}\:\sum_{r \in R}\: {}^{r}\Delta_{i}^{j} 
	      \end{align*}
	\item $C_{t}^{j}$ - defines how much execution of job $j$ cost from the beginning of the execution until time $t$.
	      Sum of all allocated resources for their time for the particular job.
	      \begin{align*}
	      	C_{t}^{j} = \sum_{i = 0}^{t}\:\sum_{r \in R}\: {}^{r}c \cdot {}^{r}x_{t}^{j} 
	      \end{align*}
	      Because $C_{t}^{j}$ is defined as sum and ${}^{r}c$ is non-negative,
	      it is true that $\forall j, t:\; C_{t+1}^{j} \geq C_{t}^{j}$.
	      Input of the program specifies maximal cost paid for job execution as a $P^j$, 
	      therefore it must be enforced by the system that this cost will not be exceeded.
	      This constraint can be defined as follows.
	      \begin{align*}
	      	\forall t, j:\; P^{j} \leq C_{t}^{j} \implies \sum_{i = t+1}^{\infty} \, \sum_{r \in R} {}^{r}x_{i}^{j} = 0 
	      \end{align*}
	      Which effectively means that when cost of job execution $C_{t}^{j}$ has reached maximal defined cost $P^{j}$,
	      no resources can be assigned to this job.
	\item $t$ - time, it is not only index but also variable, there are also constraints regarding time -
	      since client application can specify deadline to job $D^{j}$,
	      there must be additional constraint for job execution in a matter of resources assignment.
	      \begin{align*}
	      	\forall t, j:\; D^{j} \leq t \implies \sum_{i = t+1}^{\infty} \, \sum_{r \in R} {}^{r}x_{i}^{j} = 0 
	      \end{align*}
	      When maximal time is exceeded, no additional resources can be assigned to the job execution, 
	      which could be defined by following constraint.
\end{itemize}

\subsection{Resources reconfiguration}\label{subsec:resource-config}
System should be capable of changing resources assignment per job in the runtime.
This will help to distribute performance according to the current nodes load across whole network
and allow to 
Unfortunately, it is not always possible to reconfigure resources assignment while scheduling is being performed.
Therefore there must be at least one time unit, between different resources assignment,
where no resources are assigned to the job.
In other words, if resources reconfiguration is being done in time $t$ then $\sum_{r \in R} {}^{r}x_{t}^{j} = 0$.

This constraint can be defined in pseudocode,
where $m, n \in R$, $m$ are assigned resources at the time $t$,
this be written as  ${}^{m}x_{t}^{j} = 1$,
and $n$ are resources that should be assigned to the job in the time $t+1$.

\medskip

\begin{samepage}
	\begin{algorithmic}
		\IF {$ {}^{m}x_{t}^{j} = 1$} 
		\IF {$m = n$}
		\STATE ${}^{n}x_{t+1}^{j} = 1$
		\ELSE
		\STATE $\sum_{r \in R} {}^{r}x_{t+1}^{j} = 0$
		\ENDIF 
		\ENDIF 
	\end{algorithmic}        
\end{samepage}

\medskip

This constraint can defined as a mathematical function.\todo{Add mathematical definition of this constraint, is it even possible?}

\subsection{Optimization criteria}\label{subsec:optimization-criteria}
The main goal of the system is to minimize outcome value of the underlining optimization algorithm
and at the same time to minimize cost of used resources.
We can optimize single job or sum of outcomes from all jobs in the system at once.
First approach provides possibility to control and optimize outcome of a particular job,
which is an advantage for single client (owner o the job),
but it does not necessarily means that it is optimal for the whole system and vice versa.
Optimization criteria for the single job at particular time $t$ is then
maximization of weighted difference between the value enhancement reward 
and cost paid for the enhancement,
which can be described by following equation.
\begin{align*}
	\max crit_{t}^{j} = \alpha \cdot S_{r}^{j} - (1 - \alpha) \cdot C_{t}^{j} \qquad 0 \leq \alpha \leq 1 
\end{align*}
For optimization of system-wide resources and costs,
all jobs execution optimization is then defined like a
weighted sum of all rewards per jobs lowered by sum of all resources costs across the set of all jobs.
\begin{align*}
	\max crit_{t} = \alpha \sum_{j \in J} S_{r}^{j} - (1 - \alpha) \sum_{j \in J} C_{t}^{j} \qquad 0 \leq \alpha \leq 1 
\end{align*}


%%%
%% Author: lukas
%% 03.01.2019
%%

\section{Motivation to solve it}\label{sec:motivation-to-solve-it}
Here comes some motivation to solve it
\todo{Maybe bachelor degree?
Or at least unassisted project would be fine\ldots}
\todo{Lost in the context\ldots}

%TODO how we can save costs