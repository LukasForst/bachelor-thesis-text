\section{Load balancing decisions}\label{sec:load-balancing-decisions}

To make most informed load balancing decisions while scheduling multiple optimization algorithms,
the application uses dynamic scheduling with centralized node running load balancing algorithm.

The application also should take advantage of knowing the exact maximal time of execution thanks to the input parameter $D^{j}$ 
defined in section \ref{subsubsec:formal-input}.
Thanks to this parameter, 
it is possible to create scheduling decisions for much larger time horizon, 
because algorithm can know,
what workload expect in the future.

The definition formalized in section \ref{sec:formal-definition} implies 
that it is possible to use integer linear programming solver, 
because the problem itself is defined as integer linear programming problem.
That is indeed possible, 
but after careful consideration I rather decided to use heuristic approach.
The main reason for selecting this type of optimization algorithm is,
that it provides suitable results during the whole runtime.
This could be very handy when the time for load balancing decisions is tight
and in such case, the integer linear solver would not have enough time to provide suitable solution,
because it would not be optimal at all.
Also, 
it is easier to maintain and modify heuristic algorithm than the linear one
and constraint modification in the future would be easier and more flexible.

The optimization solver, I chose for the application,
was previously mentioned (\ref{subsubsec:heuristic-algs-optaplanner}) optimization engine OptaPlanner.
Unlike TASP (\ref{subsubsec:tasp}), the OptaPlanner is open sourced and can be freely used for development.
The implementation based on OptaPlanner is described in section \ref{sec:load-balancing-optaplanner}.